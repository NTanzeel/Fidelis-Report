\section{Conclusion}

The initial system was to be designed to allow users to report their bikes as stolen. This was a limited system and did not provide much potential for expansion. However, since then the project has come a long way from its first proposed objective. It has now been expanded from just a simple system for reporting bikes to a much larger system which allows users to keep track of their valuables, protects buyers from purchasing stolen goods, and aids in the recover of stolen goods. With availability across a large range of devices, the system caters for all kinds of users as opposed to the original idea of developing just a mobile app for a specific operating system. A significant amount of work has been carried out in a very limited amount of time and hence there is room for expansion, made possible by the modular approach under which the system was developed.

\subsection{Summary}

Overall, the system achieves its goals, by allowing users to sign up, register their valuables, and reporting once they've been lost or stolen. This was made possible through the use of various existing technologies, frameworks, APIs and simple forms which allowed the user to input data into the system. Through rapid prototyping, testing, and improvements based on user feedback, a successful product has been produced. The solution offered by the system is not only comparable to the existing technologies discussed in this report but in some cases better. With the decision of making a web application instead of a mobile application, the system is available across a range of devices with a modern browser. The functionality currently provided by the system is only the start of its potential and there is room for some significant improvements as well as additional features to be implemented.

\subsection{Future Work}
As previously discussed, although a functional system has been produced which meets the requirements laid out at the start of the project, there is room for significant improvements. This is due to the limited time available, rapid prototyping, and the agile methodology adopted, which aimed to produce a working version of the system rather than a perfect complete product. The future work discussed is made possible due to the modular approach which was taken when developing the system. Not only does it allow for new components to be integrated but also existing components to be replaced with improved versions.

\subsubsection{Mobile App}
The initial project required that a mobile application be developed for either iOS or Android. After some dialogue with the project supervisor it was decided that a web app would be a better approach for solving this problem as it would be accessible across all platforms. Now that a functioning system exists across platforms, a simple API could be created which would allow a mobile application to be developed for a given mobile platform. This would improve the user experience for users who plan on either solely using their mobile phone or just prefer the tailored design. A mobile app would allow for additional technologies to be employed, such as scanning the barcode of an item to retrieve all the details of an item, which are not readily available to a web application.

\subsubsection{Crime Notification}
A feature that was recommended during user testing but never implemented was the idea of alerting users in the neighbourhood whenever an item is reported as missing or stolen. This would not only alert people about the crime so they can better protect themselves but also make them aware of the item that has been stolen in case they come across it. There are several approaches that can be taken to integrate this functionality. With a mobile application, the user can simply be notified through the built in push notification system that is provided by most mobile operating systems. Within the web application, the user can subscribe to notifications in which case they would be served notification through an email. This could end up cluttering the users inbox if a separate email is sent for every report so a daily or weekly report may be sent out which concatenates all reports into one email. If the users is not subscribed to the mailing list then they can be notified through either an in-built notification system or the messaging system. 

\subsubsection{Content Reporting and Moderation}
Inappropriate content and spam content were discussed in sections \ref{Section:Social_Issues} and \ref{Section:Ethical_Issues} as some of the potential issues that may be faced by the system. The solution discussed for combatting these problems was to moderate and remove any inappropriate or spam content. This can be difficult if moderators have to browse the entire site so a possible way to speed up the process would be to implement a feature, as done so by most social media websites, which allows user to report content. The reported content would then be visible to moderators in a an administrator control panel. With this approach, an entire back end for the system would have to be built for the staff but it would utilise all users of the system as content moderators, simplifying the job required by the staff. Once content is reported, moderators can simply view the report and decided whether it is valid and take appropriate action by either removing the content or denying the report.

\subsubsection{Item History an Theft Report Generation}
This is a feature that is provided by the CheckMEND system discussed in section \ref{Section:CheckMEND}\cite{CheckMEND:Home}. Essentially, users will be able to search for an item and download a PDF which details any reports against the item, its origins and other history. This would allow users who are looking to purchase an item to perform a background check to see if the item has any significant history that could cause potential problems. The feature could also be used by users who are looking to sell their items as they could provide the report as proof to show that the item belongs to them and has no significant history. 

Simplifying insurance claims could also be achieved through this if the user was asked to provide a crime reference number upon reporting the item as lost or stolen. This would allow the system to generate a report containing all the details about the system, such as its serial no and cost, along with the crime reference which could be provided to insurance companies when filing a claim.

\subsubsection{Enhanced User Interface and Functionality}
Bootstrap has been used to provide a responsive design to users across all devices. Additionally, AJAX has been used throughout the system to load content after the document has finished loading. This speeds up initial load time by only loading essential content and then fetching the remaining content on demand. Initially the search feature was to be implemented using a similar approach, where search results are loaded through AJAX but this was changed to a standard page reload. The use of AJAX on the search page would allow further refinement of search result by providing things like a distance slider which allows user to limit the search results by distance. 

In a future version, AJAX can be implemented on various forms throughout the website where data is inserted into the system. This would allow content to be added directly without reloading the page and provide a platform for instant form validation. This approach has been used when uploading resources for an item but it is limited as it only makes state changes in the background which are not visible to the user until the page is reloaded.

\subsubsection{Matching Crime Records with Reported Items}
This again was another feature that was suggested by users throughout user testing and it has also been implemented by Stolen-Bikes discussed in section \ref{Section:Stolen-Bikes} \cite{StolenBikes:Home}. Essentially, this feature would use police data to pair up any crimes that have been reported to the police with any items that have been reported as lost or stolen by users. This can be achieved by pairing up the coordinates of the item with coordinates of the location where the crime occurred. As a result of this, users would be able to see the status of the investigation carried out by the police in regards to the crime. The pairing of crimes with police records could also be achieved through the crime reference no which would be provided by the user upon reporting the item as lost or stolen. The benefits of this functionality are vast and aid in majority of the other features that have been suggested for future revisions.

\newpage