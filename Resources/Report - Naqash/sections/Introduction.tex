\section{Introduction}
In the United Kingdom, a more economically developed country than others, we take for granted the things available to us and surround ourselves with them. This not only includes electronics, such as smartphones and tablets but also other possessions, ranging from less valuable items such as furniture to larger and more valuable items such as a bike or even a car. It is estimated that, in 2013, people in the UK collectively owned more than 606 million devices \cite{Telegraph:GadgetsPerBrit}. With a population of approximately 64.1 million at the time \cite{WorldBank:UKPopulation2013}, this means that the average British person owned 10 devices \cite{Telegraph:GadgetsPerBrit}. In the UK, where technology is an essential part of most citizens daily routine, we surround ourself with gadgets and possessions. This has created demand for not only large devices, such as desktops and TVs, to be used at home, work and other establishment but also portable devices, such as smart phones, laptops and tablets, which we can carry with us at all times. Due to this, the owners at risk of theft regardless or where they are as they can be target at their house, out on the street or in a public place.

According to Direct Line, ``the average British household contains \pounds4,000 worth of furniture, as well as \pounds3,000 worth of electrical items, \pounds2,000 worth of white goods, \pounds1,800 worth of jewellery and almost worth \pounds1,000 of curtains'' \cite{Telegraph:ContentsValue}. Additionally, a further \pounds1,200 worth of items can also be found in the garage and garden \cite{Telegraph:ContentsValue}. This adds up to a total of \pounds13,000 worth of valuables stored in an average house. These statistics are based on housing insurance and do not include the value of items not covered by housing insurance, such as a car. Tim Roots, the founder of Parago, said that ``The average working person's wardrobe contents are worth \pounds10,000'' \cite{Telegraph:ContentsValue}, which is also not covered by housing insurance and is not included in the figures by Directline, causing the total to rise significantly. Moneysupermarket.com suggests that users take a policy covering approximately \pounds40,000 \cite{Telegraph:ContentsValue}. This indicates that the average house has valuables worth around \pounds40,000 which are at risk of theft through home burglary. 

Burglary becomes an important factor when people are looking to move houses to an entirely new area and have no prior information about the neighbourhood. With the growing housing market due the mortgage tap being reopened, more and more people are looking to invest into buying their own house rather than renting \cite{BBC:HousingMarket}. According to a survey conducted by Zoopla, the average British person will move house approximately 8 times in their lifetime \cite{Zoopla:MovingStats}. This can be due to many reasons ranging from personal, economical to social. Although it is not possible to bring to light all these factors before a person moves to an area, it is possible to highlight the crime in an area due to public records. For example, when looking for a house, it would be useful to know how much burglary and theft occurs in the neighbourhood.

\subsection{Problem Definition}
Crime has been around as long as civilisation. In the 21st century, crime and especially incidents of the theft and burglary have reached an alarming level. This is due to the multitude of valuables we now carry around with us and store in our homes, as discussed above. In the 12 months prior to February 2016, 21,725 cases of theft and 70,226 cases of burglary were reported to the Metropolitan police in the London area \cite{MetPolice:CrimeFigures}. Although this is actually a decrease of 1\% and 5.2\% respectively, even if we take one item to be stolen in each of these cases, it still amounts to millions of items stolen across the various constituencies in the UK \cite{MetPolice:CrimeFigures}. It is also important to remember that not all crimes are reported to the police and hence would not be included in the figures mentioned. In fact, according to a report by the Bureau of Justice, more than half of the nations crimes and over 67\% of property crimes went unreported last year due to various reasons including fear and lack of confidence in police \cite{DOJ:UnreportedCrimes, Chronicle:UnreportedCrimes}. Another study by Stolen-Bikes suggests that 536,166 people have been a victim of bike theft in the 5 years from 2008-2013 \cite{StolenBikes:Study}. The report also states that only 1 in 4 bikes are reported after theft so if we consider 100\% of the cases then the figure rises to 2,144,664 \cite{StolenBikes:Study}.

In addition to theft, many people often end up misplacing or loosing their valuables, which usually end up in the hands of another person who may decide to return them out of good will or simply just keep them out of selfish reasons. According to a report by the TFL, 243,686 pieces of lost property were found, on the various modes of public transport and stations, either by staff or other passengers \cite{TFL:LostProperty}. This is an increase of 11.2\% from the previous year. The figures also show that only 52,106, a mere 21\% of the, items were actually restored to the original owner. Once again, we must note that this is only a small fraction of the valuables across London and an even smaller fraction of the valuables lost across the UK. If these figures from theft, burglary and lost property are counted together then the result is an enormous number of victims across the UK. To put all of this into perspective, approximately 252 were victims of theft or burglary in London, and 668 items were were reported lost to TFL, every single day of the year.

The issues discussed so far are only an introduction, leading to the real problem which is what happens to these items once they've been lost or stolen and how does it affect the victims. The term victims is used rather than victim as often there are multiple people who are affected by this, not only the original owner of the item but also the next owner of the item. Despite the best efforts by the police, only in a small number of cases the items are restored to the original owner. This is because the only way these items can be recovered is if the police has information regarding their whereabouts or come across them during another investigation. In majority of the cases, these items are sold on to local shops for quick cash or are purchased by innocent buyers who do not know about the origin of the item and its history. Once sold on, the chances of the items being recovered get slimmer and the number of victims grows with every transaction. In the rare cases that an item is recovered after it has been sold on, unless the seller choses to, the buyer of the item is never reimbursed the cost they paid to purchase the item \cite{CitizenAdvice:StolenGoods}. The buyer may follow a legal process but this just adds to the problem. It would be much simpler to check in advance, before purchasing, whether an item was lost or stolen.

\subsection{Project Motivation}
When it comes to a phone, people are willing to take initiative to get it back. While vigilantism is never a smart response, Lookout found that the vast majority of victims are willing to put themselves in considerable amount of danger to retrieve a stolen device and the personal information it carries \cite{Lookout:PhoneTheft}. According to the survey of 2,403 phone theft victims across 4 countries, 71\% of the people in the UK were willing to put themselves in some amount of danger to recover their stolen smartphone \cite{Lookout:PhoneTheft}. If people are willing to put themselves in harm just to recover their phone, it is imaginable that they would be willing to go to further extents in order to recover more valuable possessions. Loss and theft is so common that as a response to this, most portable devices such as phones and laptops can be tracked, through built-in or third-party software, as long as they are connected to the internet. Unfortunately, this alone is not enough as the items end up being sold on for quick cash before they’re connected to Wi-Fi. This provides a motivation for a system which would aid people in recovering their stolen goods without putting them in harms way. 

As discussed in the introduction, millions of British people move houses every single year and hence, it would be useful for potential movers to be able to check the level of theft and burglary in an area before they move into it. Not only this, but it would also be useful for people currently residing in the area to know when a theft or burglary has occurred in their area and explore the crime hotspots in their local area, allowing them to take greater security measures to protect their belongings, if need be. The only current solution provided for achieving this is the crime figures map that is released annually as part of a public report. The project aims to build a solution with an explore feature which allows users to explore an area and check for items reported as lost or stolen around the area, achieved by tagging each lost and theft report with the exact location. This not only greatly benefits potential movers and current residents but also aids in the restoration of them item to its original owner as there would be more people aware of the crime and the item so if they were to come across it then it could be relayed to the owner. This system would provide live and real time data to its users as soon as it is added whilst still combining the annually released data from the police.

Another motivation for such a system is to protect people from becoming victims in the first place by purchasing stolen goods. As mentioned previously, it is often the case that stolen property is sold onto pawn shops for quick cash or traded online using websites such as eBay, Amazon, and craigslist which have no safety measures in place to prevent the sale of stolen goods. There exist solutions such as CheckMEND which allow users to search through stolen phone records but these are limited and not comprehensive. A system where users can query for all items reported as lost or stolen, using just a serial number or name, before they purchase them, be it online or in a store, would save them time and money and provides motivation for them to use it. 

From personal experience, insurance companies often require that you provide details and proof of purchase of the items being reported as stolen for them to evaluate the items and reimburse the claimant. This can be difficult if the owner of the item never noted the serial number, which is stated on the item itself, before it was stolen, or if a person has owned an item for a long time and no longer has the receipt at hand. A system which allows users to privately store the details of an item upon purchase and then later retrieve them when necessary would simplify and speed up the process of insurance claims. Additionally, the user would be able to upload proof of purchase and download it later incase of warranty or insurance claims. Essentially, the system provides a way for users to log and protect all their belongings through simple means, all in one place. Simplicity is a key factor, as in a study by Google researchers found that not only will users judge websites as beautiful or not within 1/20th to 1/50th of a second, but also that “visually complex” websites are consistently rated as less beautiful than their simpler counterparts \cite{Google:SimplicityStudy}. From this it can be concluded that the simpler a system is, the more attracted users are to it and the more likely they are to use it. This again provides another motivation to build a simple system and eliminate the competition by eliminating the complexities they have introduced.

\subsection{Report}
The purpose of this report is to provide a comprehensive account of the process undertaken whilst developing the system associated with the project. The report has been broke down into 3 main sections which identify the high level stages this project went through.

\paragraph{Research and Analysis}
An introduction to the problem being faced and combatted, motivations behind the undertaking of this project and an analysis of any stakeholders is presented in chapter 1. Chapter 2 discusses and analyses any existing solutions, along with any technologies that may be used throughout the project, listing their advantages. Chapter 3 briefly discusses any issues that may arise and must be considered, whilst chapter \ref{Section:Requirements} outlines the original and final requirements for the system.

\paragraph{Development and Testing}
Chapter 5 discusses the thought process behind the designing of the system, whilst chapter 6 details the implementation of the system. Chapter 7 outlines the testing procedures that were carried out, prior to, throughout, and post development. Collectively chapters 5-7 cover the development process from start to finish.

\paragraph{Evaluation and Reflection}
The final 3 chapters reflect on the entire process of developing the system. Chapter 8 discusses the project management strategies employed to tackle the project whereas chapter 9 provides an analysis of the work carried out and how well it satisfies the initial requirements of the project. Finally, chapter 10 concludes with a summary and any suggestions for extending the system further.

\newpage