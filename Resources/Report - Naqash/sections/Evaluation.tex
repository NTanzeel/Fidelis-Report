\section{Evaluation}
Whilst the testing in section \ref{Section:Testing} identifies the adherence of the system to its functional requirements, this section provides an overall evaluation of the system against its functional and non-functional requirements based on the development and testing process. In addition to this, an evaluation of the project management strategy and the issues faced is also provided in this section

\subsection{Requirements}
In order to evaluate the system and its success, it is essential that to compare the end the result with the latest requirements for the system. Both the functional and non-functional requirements of the system were evaluated and either a pass or a fail verdict was given for each requirement. Each of the comparisons is available in the subsections below.

\subsubsection{Functional}
The table below provides a detailed analysis of the functional requirements specified in section \ref{Section:Functional_Requirements} against the finished product. This is essential in identifying whether the product has met its goals and provides the functionality set out by the project supervisor. As visible in the verdict column of the table, all the requirements were successfully met. The comments column in the table discusses whether the requirement was and how the requirement was met along with any implementation changes from the initial plan. The requirements used for comparison below are the requirements outlined in this documents, which are an adaptation of the original requirements in the project specification. Certain features were dropped as they were not feasible and hence not meeting them does not mean that the system failed the requirement.

\begin{longtable}{@{}p{0.04\textwidth}p{0.3\textwidth}p{0.1\textwidth}p{0.46\textwidth}@{}}
	\toprule
	\# & Description & Verdict & Comments \\ \midrule
	1 & Handling police data form third-party API & \textcolor{PassGreen}{PASS} & An AJAX call is used to retrieve the data from the Metropolitan Police API. The data returned by the API is in JSON format which is then decoded and the coordinates are used to plot a map displayed on the homepage. \\
	
	2 & User registration, authentication, and recovery & \textcolor{PassGreen}{PASS} & A user can sign up for system by simply providing their full name, email and password. An email address may only be used once and is designated a unique ID. The authentication process requires just the email and password. Account recovery can be achieved by entering the registration email.\\
	
	3 & Users may add locations using postcode or their current coordinates & \textcolor{PassGreen}{PASS} & Users can register a location by visiting their profile. Upon clicking the add location link, they're presented with the option to enter their postcode or use their current location via their browser. Each location is associated with the users unique ID. \\
	
	4 & Users may delete a location & \textcolor{PassGreen}{PASS} & A user can delete a location associated with their profile by simply click the delete icon on the location. The location is retained in the database for dependencies but it is hidden from the user. \\
	
	5 & Users my add items, with compulsory and optional details,  to their profile & \textcolor{PassGreen}{PASS} & A user can add an item to their profile by filling in a simple form. Certain fields are required and the item cannot be saved with out them, Additional data may be entered optionally. Upon saving, the item is made private and only visible to the owner. \\
	
	6 & Users may edit previously saved items & \textcolor{PassGreen}{PASS} & In order to edit an item, the user can simply click the edit button which populates the create form with the saved data. All details except the cover image can be changed through the edit page. \\
	
	7 & Users may delete an item belonging to them & \textcolor{PassGreen}{PASS} & As with locations, users can delete an item by clicking the delete button. Once again, these details are retained in the database but hidden from the user. \\
	
	8 & An item can be reported as lost or stolen & \textcolor{PassGreen}{PASS} & An item can be reported as lost or stolen by clicking edit item. Once on the edit page, the user can report the item by providing the location of theft on the report tab. This makes certain details available on the public search. \\
	
	9 & An item can be marked as recovered & \textcolor{PassGreen}{PASS} & An item can be marked as recovered by clicking edit item. Once on the edit page, the user can mark the item as recovered by clicking a simple button on the report tab. This removes the item from public search by soft deleting the report. \\
	
	10 & The user may upload resources associated to one item & \textcolor{PassGreen}{PASS} & The user can upload a cover image on the create item page. Additional resources can be uploaded by visiting the edit item page and click the resources tab. This displays all the resources and clicking the add button provides a modal for uploading new resources. All resources are private but the privacy of images can be toggle by clicking edit on a resource. \\
	
	11 & The user may edit an existing resource & PASS & Once a resource has been upload the user can click the edit button on the resource to bring up a modal with all the properties that can be changed. By default the alias of any resource can be changed to help the user identify it by name. Additionally, for all images except the last remaining public image, a private and public toggle is available. \\
	
	12 & The user may delete a resource & \textcolor{PassGreen}{PASS} & As with items and locations, a resource can be deleted by clicking the trashcan icon as long as there is at least one public resource. Upon deleting an item, the resource entry in the database is soft deleted but the resource is still kept in storage. It is simply made unavailable. \\
	
	13 & Search & \textcolor{PassGreen}{PASS} & Any user of the system, registered or not, can use the search feature on the system. The search feature allows searching for items through any of the textual fields. The results are presented in a grid view and have default ordering. \\
	
	14 & Explore around the users current location & \textcolor{PassGreen}{PASS} & The user can explore around their current location by clicking the around me link in the navigation bar. This required location access from the user. Te results are displayed in grid and map view but the map view is hidden on mobile devices. The results can also be ordered by several attributes. \\
	
	15 & The user may view details about an item & \textcolor{PassGreen}{PASS} & Details about an item can be viewed by clicking one of the results on the search or around me page. Only certain details are made public for all users to see when an item is reported. There is no direct link to viewing the items details, due to privacy. Registered users can also message the owner through this popup. \\
	
	16 & Registered users may message through an in-built messaging system & \textcolor{PassGreen}{PASS} & An in-built email like messaging system is available. However the user may only use this to message the owner of an item that has been reported stolen. Once the item has been marked as recovered the user will no longer be able to message the owner or vice versa. \\
	
	17 & In-built messaging system support & \textcolor{PassGreen}{PASS} & As discussed above, an in-built messaging system is available and allows communication between users in a format similar to emailing. This eliminates the need for users to exchange personal contact information unless they wish to. \\
	
	18 & Recipients may reply to messages & \textcolor{PassGreen}{PASS} & Upon receiving a message, user can click a reply link to respond to messages. These messages are store separately in a sent folder whereas received messages are stored in an inbox folder. \\
	
	19 & Senders or recipient may delete messages & \textcolor{PassGreen}{PASS} & A user can delete a message by clicking the trashcan icon next to a message. This does not hard or soft delete the message as this would hide it from both users. Instead another record is inserted to identify which user has deleted the message. \\
	
	20 & All data must be related to an individual & \textcolor{PassGreen}{PASS} & Any record in the database is either directly associated to a user through a User ID or indirectly linked to a user through pivoting tables. For this reason no records are ever deleted. \\ \bottomrule

	\caption{Evaluation of functional requirements}
	\label{table:Functional_Evaluation}
\end{longtable}

\subsubsection{Non-Functional}

Non-functional requirements ensure that the developed solution is applicable to usage in the real word. Functional requirements may ensure that the system provides all the features required, but these could be entirely useless if the system is not user friendly as no one would use it. Table \ref{table:Non-Functional_Evaluation} discusses all the non-functional requirements and how closely these requirements were met. The comments column discusses how the requirement was met and whether any aspects of the requirement were missed. 

As visible in the table, the only requirement that was not fully satisfied and hence didn't receive a pass was requirement 4 regarding scalability. This is because although all the decision were made with storage and other resources in mind, some of the decisions were made due to the nature of the service and could not be compromised to meet this requirement. This is especially true for the uploading of resources, which is a necessary part of the system but uses up a significant amount of storage space. This was optimised as far as possible but at a point the hardware will not be sufficient and will have to be expanded. In addition, as the system grows, having one server will not be sufficient enough to deal with the load and the network will need to be expanded.

\begin{longtable}{@{}p{0.04\textwidth}p{0.2\textwidth}p{0.1\textwidth}p{0.56\textwidth}@{}}
	\toprule
	\# & Description & Verdict & Comments \\ \midrule
	1 & Compatibility & \textcolor{PassGreen}{PASS} & The system has been tested across several modern browsers including Safari, Chrome and Safari on iOS. The use of Bootstrap has allowed for a responsive design across all devices. Functionality is not limited on any of these devices but in some cases, features are hidden. \\
	2 & Usability & \textcolor{PassGreen}{PASS} & The system passes this requirement as users were given a list of tasks to complete without any guidance and they managed to do so using their own intuition. A consistent layout is provided across the website through the use of master views which maintain headers across all pages. \\
	3 & Security & \textcolor{PassGreen}{PASS} & Security was achieved on user credentials by hashing the user passwords. The database has also been password protected making it harder to breach. Resources have been guarded by generating random names for files and directories. However, if the user has a link to a resource, they will always be able to access it. \\
	4 & Scalability & \textcolor{orange}{-} & Whilst all decisions taken have been taken with storage and processing power in mind, larger datasets may result in slow response from the system on the current hardware. Due to the fact that users are allowed to upload resources, it is difficult to contain storage within a limit and hardware upgrades will be necessary as the system grows. It is also impossible to test this requirement without a large user base. \\
	5 & Extensibility & \textcolor{PassGreen}{PASS} & The system was developed using a modular approach made possible through the use of an MVC approach. This means that the system can be expanded at any time as each module is completely independent of another and can be upgraded or replaced with ease. This also allows new functionality to be integrated into the system with ease. \\
	6 & Maintainability & \textcolor{PassGreen}{PASS} & The system is made maintainable by the documentation on all of the PHP code written throughout the project. Good coding practices as laid out by the frameworks and PHP manual were used. In addition, version control was used with through details provided for each commit. A regular update was performed for the Laravel framework to ensure that the latest version was being used. All third party apis use a CDN which means they're updated as soon as new version become available. \\
	7 & Readability & \textcolor{PassGreen}{PASS} & As mentioned in maintainability, the code has been well documented. This is done only for the PHP code as the remaining code is relatively self explanatory. All changes were logged through version control as well as other progress tracking tools and a clear outline is provided in this report. \\
	\caption{Evaluation of non-functional requirements}
	\label{table:Non-Functional_Evaluation}
\end{longtable}

\subsection{Legal, Ethical, Social and Professional Issues}

\subsubsection{Legal Issues}
During the sign up process, the user is requested to provide their full name, email and password. These are clearly labelled on the form and the user is made aware that these details will be stored to authenticate the user in the future. In addition to this, these details are displayed in the users dashboard so they have knowledge of the fact these details are still being retained. In addition to this, the users sessions and connection details are also stored in a cookie on their computer. The user is not made aware of this as it does not store any personal information, the cookie is used purely to remember the user until they explicitly logout. The details stored in the cookie can be removed at any time by using the logout link which removes any details stored about the user outside of the database. Any sensitive data such as the users password is being hashed and cannot be unencrypted in reasonable time, even if it is compromised.

Additional data that is input by the user into the system, such as their location data and items is also stored in the system. Upon saving these details, the users receives a notification letting them know that the data has been saved. Although this is not explicitly letting the user know how this data is being used, it still makes it clear that the data is being held. The user can then view all of this information in their dashboard and remove at any time. The data will still be held but it will not be visible to anyone unless they have access to the database. The user may also request a permanent delation. This additional data is associated with the user through their ID so even if this additional data is compromised, there would be no way of identifying the user it is associated with. As the data being stored is not sensitive, such as bank card details or address, and cannot be used to identify the individual, maintaining it is not as much of an issue. The user may also request that any data stored about them is revoked. Although this is not facilitated as a feature in the system, the user can notify the administrator and their request can be dealt with accordingly.

Resources uploaded by users were another legal concern from a licensing view but also a security view. The resources uploaded by users are only available to the uploader and secured from other users by generating random directory names and filenames. Several directories are created within one another to build a tree structure which cannot be explored by users. This means that the only way to access a resource is through the full path generated upon runtime and stored in the database. In order for licensing to become an issue, these resource must first be made public but in the case that this problem does arise, original authors of the resource can contact the system administrators and ask for it to be removed. This should mostly not be an issue as the resources uploaded by users aren't being used for commercial but only for personal use which is generally not prohibited.

\subsubsection{Social Issues}
One of the major issues that were discussed was the systems reliance on user providing appropriate data. This meant that users could flood the system with spam data, impacting the reliability of the system and deterring users. Although there is no official solution to prevent this entirely, validation has been added across the entire system. The validation ensures that users must provide data in a certain format and cannot leave fields blank, making the procedure of spamming the system slightly more complicated and longer. Additional security features were implemented through the use of single case form tokens which prevents user from writing scripts to submit false data. The forms can only be submitted to the backend through the actual system.

The secondary issue brought up was the availability of personal information about one user to another user. If personal information were to be available about one user to other users, this could lead to potential issues violating privacy of users. As a result of this, minimal personal details were stored to start with. Any information stored about the user was kept in a single table and each user was assigned a unique id which was used to refer to users throughout the system. This allowed data to be associated with a user but not necessarily a user to be associated with data if someone just had access to the additional data such as items. In rare cases where the users details are made public, it is for the safety of other users. The only details ever made public are the name of a user and the postcode where they've reported an item to be stolen. Even then, the location is generalised to a postcode meaning there is no way to identify the users exact location and the name is truncated, making it difficult to identify users.

These are just some of the measures that have been put in place which reduce the chances of these issues arising, but do not completely prevent them. There is in reality, no way to guarantee 100\% security and hence a reasonable level of security has been provided here. Additional features, such as administration systems, will be suggested in the conclusion as future work which help the system combat these issues at a larger scale and with ease.

\subsubsection{Ethical Issues}
It is almost impossible to prevent users from uploading explicit or inappropriate content whilst still allowing them to upload images as resources. The chances of such content being made public have been reduced by preventing resources such as videos and files from being made public. This means the only form of inappropriate content that can still be uploaded is images but this cannot be prevented without completely disabling public resources. However, users can report content like this to the administrators to have it removed. The issues regarding the users privacy has been covered as a legal and social issues in the sections above.

\subsubsection{Professional Issues}
The user privacy has been considered throughout the development process and as a result, minimal information has been available to the users. In all cases, users data is hidden from other users unless it is impossible to implement the required functionality without compromising. As previously discussed, the users credentials have been stored as a hash of the original data meaning that no one can access the original data. This is true for moderators, administrators and even the developers of the system. The users personal data is never made available to any third party services as the only information ever exchanged is the postcode entered by a user during the decoding process using the Google Maps API. In no case will the users credentials ever be compromised, even if someone gains unauthorised access to the data.

Thorough care has been taken during development to ensure that professional coding standards were adhered to. These standards were available in the framework documentation by Laravel and in the PHP manual \cite{Laravel:Home, PHP:Home}. These were achieved by the use of an IDE which supports documentation through simple shortcuts and strongly enforces the consistency of documentation with the code. Not only this, progress logging tools were used along with this report to keep track of changes and provide a through log of the development process.

\subsection{Project Management}
Section \ref{Section:Project_Management} of this report discusses how project management is a key aspect of any successful project as well as how project management was achieved in this project. In order for the project to be successful, dedication from the developer, support from the supervisor and all stakeholders, and users willing to aid in the development of the system through user testing and feedback were required. Along with these base requirements, a solid schedule was also required to ensure timely delivery of the final product. This schedule was developed as part of the original project specification and is available in figure \ref{fig:GanttChart} but it has since been adjusted, as discussed in the progress report.

The agile methodology adopted for the development of this project proved to be highly useful. Not only did the agile approach accommodate the changing requirements that emerged, but it also allowed user feedback to be incorporated through the development process by developing functional version and then refining these based on the feedback. The changes made throughout the project were discussed with the project supervisor during meetings and other stakeholders, before they were incorporated into the system.

Several progress management tools were also employed to keep track of the progress and guide the project. Tools such as Trello allowed requirements to be split up into small tasks which could easily be implemented and ticked off. These tools were particularly helpful in adhering to the timeline and allowed the development time to be focused on the larger priority tasks.

\newpage