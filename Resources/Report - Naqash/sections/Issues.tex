\section{Legal, Social, Ethical and Professional Issues}
When developing a system that requires users to input personal data, it is important to consider the all the challenges that may be associated with this. Mishandling personal data about a user can raise a large number of issues. These issues are split up into legal, social, ethical, and professional issues. Each category has been discussed further and details specific cases that need to be dealt with for that category, when designing the system.

\subsection{Legal Issues} \label{Section:Legal_Issues}
The system requires that users provide personal information about themselves in order to take full advantage of the features available, a requirement introduced by the nature of the services being offered. In order to do this, the system must comply with the rules and regulations set in place for handling sensitive data, and data in general. These rules and regulations vary across countries but as the system is primarily being developed for the UK, only the laws set out in the UK will be considered.

In addition to the handling of data, laws regarding the access to resources must also be considered. This is applicable when 3rd-party APIs are being used to provide a service but also when it comes to using media available on the web. As such, the licensing term set out by the distributor must also be adhered to. This will be discussed further in the licensing section.

\subsubsection{Sensitive Data}
Within the UK, the Data Protection Act (1998), an Act to make new provision for the regulation of the processing of information relating to individuals, regulates the use of sensitive and personal information \cite{Legislation:DataProtectionAct}. As per the Data Protection Act, it must be made clear to the user what information is being requested and stored. Under the act, the user is also able to request that any data held about them is removed from the system if it is no longer necessary. Although this may not be a specific feature of the system, the vendor must provide this option, even if it is through a written or verbal request.

\subsubsection{Licensing}
As the system uses various APIs, such as Google Maps, it is necessary that the licensing terms for these set out by the vendor are adhered to. Some vendors require that their trademark be placed wherever their services are used and this must be adhered to strictly when developing the system. Resources, taken from the web, such as images and snippets, will be used throughout the system and as such credit needs to be given when required by the author in order to avoid copyright infringement.

\subsubsection{Resources}
Due to the upload functionality, users will be able to upload content onto the servers which may not belong to them and infringes copyright. This must be taken into account and if a copyright claim is made then the uploader must be notified and the content must be removed immediately to avoid legal issues.

\subsection{Social Issues} \label{Section:Social_Issues}
The data output is input entirely by the users of the system, and hence the system is completely reliant on users providing appropriate data. This can leads to issues as users could flood the system with spam or inaccurate data. An example of this is that criminals could sign up to the system and flood the data in an area by reporting a large number of fake items as lost or stolen. This would result in actual reports being surrounded by spam, making it difficult for the users to actually identify which are genuine and which are false. There are several ways in which these issues can be reduced, including but not limited to content moderation and content allowance, all of which must be considered during development.

Another issues that must also be dealt with is the availability of personal information about one user to another. The details of all users must be hidden from all other users and must be completely inaccessible until, the user reports an item as lost or stolen and, willingly makes certain data, such as their name, available. Once the user makes their data available, other users may contact them through the in-built messaging system only and no other contact details should be exchanged.

\subsection{Ethical Issues} \label{Section:Ethical_Issues}
Once again, due to the unrestricted nature of the system and the upload functionality, users may be able to upload explicit, inappropriate, or restricted content \cite{SocialNetworkSites:RisksAndBenefits}. Although this content will initially be private, it could be made available to the public and users could unexpectedly encounter it. This could deter the users from using the system and may result in a loss of users \cite{SocialNetworkSites:RisksAndBenefits}. As such this issue must be considered during development and measures should be put in place to limit it, be it through content restriction or by regularly moderating the new information added to the system.

Along with being raised as a legal issue, the users privacy is reiterated as an ethical issue. This must be considered throughout the development process and all measures should be taken to ensure that the users privacy is respected by protecting their data.

\subsection{Professional Issues} \label{Section:Professional_Issues}
Protecting the users data and their privacy is part of the professional service being offered. Thus security is also a professional issue and needs to be considered when the service is deployed. It is necessary to protect the users credentials, as well as any additional data entered by the user. This means security measures will need to be taken on both the software and hardware side. The users credentials must be encrypted so if they are to be stolen, their privacy is still not breached. 

When and if the system is being moderated, the personal data of the user should not be available to the moderators or any 3rd-parties involved in the process. This applies under all conditions, unless a user has violated the terms \& conditions of the system, in which case they forfeit their right of privacy. This is because their account details may need to be examined as part of the moderation and quality assurance process.

In addition to the aforementioned issues, it is the duty of the developer to follow the professional coding standards. This will allow the system to be developed beyond the initial plan by either the same team or another team of developers. The code must come with detailed documentation in the form of an overview as provided by this report and in the form of commenting. As previously discussed, any additional media or resources used from third parties must be appropriately acknowledged, both on the webpage and in this report.

\newpage