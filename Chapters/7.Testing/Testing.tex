\chapter{Testing}
\label{Chapter:Testing}

\section{Unit Testing}
Unit testing is a software development process in which the smallest testable parts of an application, called units, are individually and independently scrutinised for proper operation \cite{TechTarget:UnitTesting}. Through unit testing, the developer can test each aspect of the system at a micro level before each component is integrated into the system. These unit tests were mostly dictated by the functional and non-functional requirements of the system to ensure that the component satisfied all the requirements. Throughout this process, white-box testing has been used along with some black-box testing to ensure expected behaviour is provided by the system. Table \ref{tab:unit-testing} below discusses the tests carried out as part of unit testing by stating the expected behaviour and the actual behaviour.

\begin{longtabu} to \textwidth {XXXX}
\hline
Requirement & Description & Comment & Result \\ 
\hline
\textbf{F2} & Valid login & Login form is filled with correct email and password & \textcolor{passgreen}{PASS} The user is authenticated and redirected to the logged-in home page \vspace{2mm}\\
\textbf{F2} & Invalid login & Login form is filled with incorrect email and password & \textcolor{passgreen}{PASS} The user is not logged in and the error is shown to the user on the login page \vspace{2mm}\\
\textbf{F2} & Logging out & `Log out' is selected from the navigation bar & \textcolor{passgreen}{PASS} The user is no longer authenticated and is relocated to the guest home page \vspace{2mm}\\
\textbf{F2} & User can register & The registration form is filled in by the user & \textcolor{passgreen}{PASS} New account is added to the database and the user is redirected to the logged-in home page \vspace{2mm}\\
\textbf{F2} & Email is associated with more than one account & The registration form is filled in with a pre-existing email & \textcolor{passgreen}{PASS} The account is not added and the error is highlighted to the user on the registration page \vspace{2mm}\\
\textbf{F2} & Username is associated with more than one account & The registration form is filled in with a pre-existing username & \textcolor{passgreen}{PASS} The account is not added and the error is highlighted to the user on the registration page \vspace{2mm}\\
\textbf{F2} & All registration fields are filled & The registration form is filled with empty fields & \textcolor{passgreen}{PASS} The account is not added and the error is highlighted to the user on the registration page \vspace{2mm}\\
\textbf{F2} & Matching password and confirmation & A confirmation is entered which differs from the password & \textcolor{passgreen}{PASS} The account is not added and the error is highlighted to the user on the registration page \vspace{2mm}\\
\textbf{F2} & The password is at least 6 characters long & A password shorter than 5 is entered during registration & \textcolor{passgreen}{PASS} The account is not added and the error is highlighted to the user on the registration page \vspace{2mm}\\
\textbf{F2} & The user is at least 13 years old & The date of birth is set such that the user is younger than 13 years old & \textcolor{passgreen}{PASS} The account is not added and the error is highlighted to the user on the registration page \vspace{2mm}\\
\textbf{F2} & Reset password & User's email is entered on the forgotten password page & \textcolor{passgreen}{PASS} User receives an email with a link to reset their password and that password replaces the existing one for that account \vspace{2mm}\\
\textbf{F2} & Delete account & The `Delete Account' button is clicked on the settings page & \textcolor{passgreen}{PASS} After confirming the user is sure about deleting their account, the account is soft deleted and their content is no longer visible to other users \vspace{2mm}\\
\textbf{F3} & Make a public account private & The `Private Account' option on the profile settings page is toggled on & \textcolor{passgreen}{PASS} User content becomes hidden to other users unless they are pre-existing followers \vspace{2mm}\\
\textbf{F3} & Make a private account public & The `Private Account' option on the profile settings page is toggled off & \textcolor{passgreen}{PASS} User content becomes visible to other users, even if they aren't followers \vspace{2mm}\\
\textbf{F3} & Following a private account & A user clicks `Follow' on a private account's profile & \textcolor{passgreen}{PASS} The private account receives a notification asking for permission for the user to follow them and, until the user is granted permission, the user can not view the private user's content \vspace{2mm}\\
\textbf{F4} & Blocking a user which is not a follower & A user clicks `Block' on a user's profile who is not a follower & \textcolor{passgreen}{PASS} The blocked user is added to the other's block list and the user's content is not longer visible to the blocked user, including the profile page \vspace{2mm}\\
\textbf{F4} & Blocking a user which is a follower & A user clicks `Block' on a follower's profile & \textcolor{passgreen}{PASS} The blocked user unfollows the other user, the blocked user is added to the other's block list and the user's content is not longer visible to the blocked user, including the profile page \vspace{2mm}\\
\textbf{F4} & Unblocking a user & A user clicks `Unblock' on a blocked user's profile & \textcolor{passgreen}{PASS} The newly unblocked user is able to view the other user's content \vspace{2mm}\\
\textbf{F4} & Unblocking a user & A user clicks `Unblock' on a blocked user's profile & \textcolor{passgreen}{PASS} The newly unblocked user is able to view the other user's content \vspace{2mm}\\
\textbf{F5} & Following a non-private account & A user clicks `Follow' on a non-private user's profile & \textcolor{passgreen}{PASS} The followed user will be added to the other user's following list, and their content will become visible on the following user's home feed \vspace{2mm}\\
\textbf{F5} & Unfollowing an account & A user clicks `Unfollow' on a user's profile & \textcolor{passgreen}{PASS} The unfollowed user's content no longer appears on the other user's home feed and the unfollowed user is removed from the other's following list \vspace{2mm}\\
\textbf{F6} & Viewing a category feed & A user clicks on a category on the discover page & \textcolor{passgreen}{PASS} The user is redirected to that category's feed, which displays posts belonging to that category \vspace{2mm}\\
\textbf{F6} & Viewing a tag feed & A user clicks on a tag from a post & \textcolor{passgreen}{PASS} The user is redirected to that tag's feed, which displays posts containing that tag \vspace{2mm}\\
\textbf{F6} & Subscribing to a tag & A user clicks on a the `Subscribe' button on an unsubscribed tag's discover page & \textcolor{passgreen}{PASS} The tag is added to the user's list of subscribed tags and content containing that tag appears in the user's subscribed feed \vspace{2mm}\\
\textbf{F6} & Unsubscribing to a tag & A user clicks on a the `Unubscribe' button on a subscribed tag's discover page & \textcolor{passgreen}{PASS} The tag is removed from the user's list of subscribed tags and content containing that tag no longer appears in the user's subscribed feed \vspace{2mm}\\
\textbf{F7} & Viewing home feed & A user navigates to the home page & \textcolor{passgreen}{PASS} The home feed displays content from their followers, which is prioritised by the reputation score of the content and user \vspace{2mm}\\
\textbf{F8} & Posting text & A user creates a new post containing text only & \textcolor{passgreen}{PASS} The post is displayed on the user's and their followers' feeds and, if any words in the post correspond to tag in the tags table, the post is assigned that tag \vspace{2mm}\\
\textbf{F8} & Posting an image & A user creates a new post containing an image & \textcolor{passgreen}{PASS} The post and image are displayed on the user's and their followers' feeds \vspace{2mm}\\
\textbf{F8} & Posting multiple images & A user creates a new post containing multiple images & \textcolor{passgreen}{PASS} The post and images are displayed on the user's and their followers' feeds \vspace{2mm}\\
\textbf{F8} & Posting tags & A user creates a new post containing a tag (using the \# symbol) & \textcolor{passgreen}{PASS} The post is displayed on the user's and their followers' feeds and the tag serves as a link to the tag's feed on the discover page \vspace{2mm}\\
\textbf{F8} & Categorising tags & A user creates a new post containing a tag (using the \# symbol) and the categorisation Python script is run & \textcolor{passgreen}{PASS} The tag is assigned a category, either via the category key word bin or the categorisation classifier, and posts containing that tag are displayed on that category's discover feed \vspace{2mm}\\
\textbf{F8} & Categorising posts & A user creates a new post containing text & \textcolor{passgreen}{PASS} The post is assigned a category and is displayed on that category's discover feed \vspace{2mm}\\
\textbf{F8} & Editing post category & A user selects a new category for the post in the edit post modal & \textcolor{passgreen}{PASS} The post is assigned to the new category and is displayed on that category's discover feed \vspace{2mm}\\
\textbf{F8} & Removing post category & A user selects the `No category' option for a post in the edit post modal & \textcolor{passgreen}{PASS} The post's category is removed and is no longer displayed on the previous category's discover feed \vspace{2mm}\\
\textbf{F8} & Hiding a post & A user clicks the bin icon on one of their own posts & \textcolor{passgreen}{PASS} The post is hidden from the user and the content's owner's reputation score is reduced \vspace{2mm}\\
\textbf{F9} & Viewing post & A user clicks on a post containing text & \textcolor{passgreen}{PASS} The user is redirected to the post's viewing page \vspace{2mm}\\
\textbf{F9} & Viewing image & A user clicks on an image from a post & \textcolor{passgreen}{PASS} A modal is opened displaying the image and, if the post contains multiple images, arrows to allow scrolling \vspace{2mm}\\
\textbf{F9} & Viewing post of private user who is not being followed & Enter a private post's via the URL & \textcolor{passgreen}{PASS} The browser serves a 401 notice \vspace{2mm}\\
\textbf{F9} & Scrolling images & Click the left and right arrows when viewing an image belonging to a post with multiple images & \textcolor{passgreen}{PASS} The next/previous image from the post is displayed \vspace{2mm}\\
\textbf{F10} & Commenting on a post & The user types a comment on another user's post & \textcolor{passgreen}{PASS} The comment is appended to the post and the post's owner is notified of the comment \vspace{2mm}\\
\textbf{F11} & Up-voting another user's post & The user clicks the thumbs up icon on another user's post & \textcolor{passgreen}{PASS} The number of up-votes for the post is incremented \vspace{2mm}\\
\textbf{F11} & Undoing an up-vote another user's post & The user clicks the thumbs up icon on another user's post, which had previously been up-voted & \textcolor{passgreen}{PASS} The number of up-votes for the post is decremented \vspace{2mm}\\
\textbf{F11} & Down-voting another user's post & The user clicks the thumbs down icon on another user's post & \textcolor{passgreen}{PASS} The number of down-votes for the post is incremented \vspace{2mm}\\
\textbf{F11} & Undoing a down-vote of another user's post & The user clicks the thumbs down icon on another user's post, which had previously been down-voted & \textcolor{passgreen}{PASS} The number of down-votes for the post is decremented and the reputation of the content is restored to the previous level \vspace{2mm}\\
\textbf{F11} & Giving feedback on own post & The user clicks the thumbs up/down icon on their own post & \textcolor{passgreen}{PASS} No change is made to the number of up-votes or down-votes for the post \vspace{2mm}\\
\textbf{F12} & Notification for reply to a post & A user comments on another user's post & \textcolor{passgreen}{PASS} The owner of the post gets a notification saying the user has commented on their post \vspace{2mm}\\
\textbf{F12} & Notification for vote on a post & A user votes on another user's post & \textcolor{passgreen}{PASS} The owner of the post gets a notification saying the user has voted on their post \vspace{2mm}\\
\textbf{F12} & Notification for a user following an account & A user clicks `Follow' on another user's profile & \textcolor{passgreen}{PASS} The user who is being followed is given a notification \vspace{2mm}\\
\textbf{F13} & Changing profile picture & A user uploads a new profile picture & \textcolor{passgreen}{PASS} The new profile picture replaces the old one throughout the site \vspace{2mm}\\
\textbf{F13} & Changing cover picture & A user uploads a new cover picture & \textcolor{passgreen}{PASS} The new cover picture replaces the old one throughout the site \vspace{2mm}\\
\textbf{F14} & Calculating user reputation score & Run user reputation Python script & \textcolor{passgreen}{PASS} Each user's reputation is calculated based on the number of up-votes, down-votes, comments, posts, abusive posts and followers \vspace{2mm}\\
\textbf{F15} & Calculating post reputation score & Run post reputation Python script & \textcolor{passgreen}{PASS} Each post's reputation is calculated based on the number of up-votes, down-votes and comments \vspace{2mm}\\
\textbf{F16} & Changing reputation settings & The user changes the reputation settings in the settings page &  \textcolor{passgreen}{PASS} The reputations displayed to the user are adapted with regards to the new reputation settings \vspace{2mm}\\
\textbf{F17} & Recommending users & Run recommendation Python script and view recommendation widget & \textcolor{passgreen}{PASS} The users recommended are prioritised by the reputation score as well as the factors which are stipulated by the currently selected recommendation method (explorer, friend-of-a-friend or hybrid) \vspace{2mm}\\
\textbf{F17} & Recommending posts & Run recommendation Python script and view `My recommendations' discover page & \textcolor{passgreen}{PASS} The posts recommended are prioritised by the reputation score as well as the factors which are stipulated by the currently selected recommendation method (explorer, friend-of-a-friend or hybrid) \vspace{2mm}\\
\textbf{F18} & Changing recommendation settings & The user changes the recommendation settings in the settings page & \textcolor{passgreen}{PASS} The recommendations made to the user are adapted with regards to the new recommendation settings\vspace{2mm}\\
\textbf{F19} & Detecting abuse & Run abuse detection Python script and view home page & \textcolor{passgreen}{PASS} Posts are hidden if they are deemed offensive by the abuse detection algorithm \vspace{2mm}\\
\textbf{F19} & Changing abuse detection threshold & The user changes the abuse detection threshold in the settings page & \textcolor{passgreen}{PASS} The posts which are hidden are altered depending on the new threshold set \vspace{2mm}\\
\textbf{F19} & Viewing flagged content & The user selects `Toggle View' option on a flagged post & \textcolor{passgreen}{PASS} The content of the post is displayed \vspace{2mm}\\
\textbf{F20} & Reporting content & The user selects the flag icon on another user's post & \textcolor{passgreen}{PASS} The post was removed from the user's feed \vspace{2mm}\\
\hline

\caption{Unit testing}
\label{tab:unit-testing}
\end{longtabu}

\section{Integration Testing}
Integration testing is a logical extension of unit testing. In its simplest form, two units that have already been tested are combined into a component and the interface between them is tested \cite{MSDN:IntegrationTesting}. As the major functionality of each component was completed, the component could now be integrated into the system so that it may work alongside the other components. Once integrated, integration testing could be carried out to ensure that each individual component functioned alongside other components, that it may have relied on, without resulting in errors. Integration testing prevents errors from propagating into the subsequently implemented components. Every time a component was integrated, all the unit tests for that component were run again to verify its functionality. In addition, unit tests for other components which may have been impacted by the integration were also run again. Any errors that occurred were fixed and the tests were rerun before moving onto the next component.

\section{System Testing}
System testing is most often the final test performed on a system to verify that it meets the specification and works as expected \cite{ISTQB:SystemTesting}. In system testing the behaviour of whole system is tested as defined by the scope of the project or product \cite{ISTQB:SystemTesting}. However, system testing was carried out for the project as the system was developed and at the end. As each component was developed and integrated, thorough testing was done on the component, and any other components it affected, including general system functionality. In comparison to unit and integration testing which verify that functional requirements are met, system testing verifies both functional and non-functional requirements of the system. Additionally, at the end of the development process, a comprehensive and thorough system test was carried out using multiple user accounts. These tests followed the same convention and produced the same results as the unit tests. Any errors that occurred were patched on the go and the testing was done all over again.

Testing at times can be cumbersome, and to reduce the work needed from developers cron jobs were scheduled that emailed error logs to all developers. 

\subsection{Validation Testing}
Validation testing was performed to ensure that validation rules created for all requests worked as expected. This form of testing involved verifying validation on user forms. Inputs for all forms used throughout the system were tested by providing blank or incorrect input values to check that erroneous forms were rejected, and the appropriate error messages displayed to the user. Visually informing the user of an error in their input is vital to their interactivity with the system as it provides them with feedback on their actions. Testing forms involved providing valid, invalid and boundary inputs (i.e. just one field filled). Custom requests were defined, and also required testing.

\begin{figure}[H]
\centering
\includegraphics[height=2in]{Images/Testing/BlockRequest}
\caption{Custom request for blocking users}
\label{fig:BlockRequest}
\end{figure}

Figure \ref{fig:BlockRequest} shows an example of one of the custom requests used for blocking a user. In this we can see two functions: one for checking the authorisation of the user making the request and another which includes a set of validation rules. 

\subsection{Permission Testing}
Fidelis is split into accessible pages for authorised and unauthorised users. As such, permissions for page access and actions permissible for a user needed to be tested. This form of testing is important as it deals with the protection of users and their data. The first portion of this testing involved looking at pages that should only be accessible to authorised users. Functionality for checking user authorisation was done by the authentication middleware, but this was tested to make sure it was applied correctly. The remainder of permission testing dealt with verifying requests on a per user basis. This involved checking that a user was permitted to perform the operation they were attempting. For example, user $B$ should not be able to edit or delete posts made by user $A$. Additionally, only user $A$ should be able to modify their settings.

\section{User Acceptance Testing}
User Acceptance Testing (UAT) is one of the last stages of the software development lifecycle, and is performed on the end-user of a given product to gain feedback and approcal on the final product \cite{EconomicTimes:UAT}. UAT is an important part of overall testing and from it, it is possible to identify bugs and gain useful insights into the system and how users view it. This section will look at how UAT was conducted for this project, and in particular will focus on the feedback received from users.

\subsection{User Feedback}
The entire system relies on a stable and growing user base; without the users the system would be rendered useless. With this in mind, guaranteeing a more than satisfactory user experience was an integral to system success, and it was therefore important to gain feedback on the user experience offered by Fidelis. By correctly analysing feedback received from users, system strengths and weaknesses could be identified for each progressive system prototype. A number of users were consulted after being given access to the system at various stages of development in order to identify any issues that users may have with the system so that these could be dealt with in latter development stages.

Throughout the development of the system, users were given questionnaires along with a list of tasks to complete. Once the users completed each task, they would answer the questions corresponding to the tasks. No guidance was given to the user and they were expected to use their intuition to navigate the system and perform the tasks. The survey not only included questions on system navigation, but also had questions related to UI design, recommendations and settings provided for privacy and security. Figure \ref{fig:UATQuestionnaire} shows part of the survey given to users after development was completed. The survey accepted a combination of multiple choice and written answers. Multiple choice questions were used to gauge overall user consensus, whereas as written answers provided more detailed feedback on specific system components.

\begin{figure}[H]
\centering
\includegraphics[height=3in]{Images/Testing/UATQuestionnaire}
\caption{Section from final questionnaire given to users as part of UAT}
\label{fig:UATQuestionnaire}
\end{figure}

As mentioned previously, Fidelis design adopted a lot of design elements from Twitter. As a result of this, users felt very comfortable navigating through and using Fidelis. Feedback received, mostly orally, highlighted that the system did need to re-design elements of the UI as it felt and looked too similar to Twitter. A discussion of UI re-design is included in Chapter \ref{Chapter:Conclusion}. Users were very receptive towards the idea of controlling how recommendations were generated for them. By being able to specify a method for recommendations, users were enabled to buy into the idea of Fidelis being a platform that enables them to engage with content they are interested in and have control over. Similar sentiments were voiced on the ability to fine-tune how abuse detection and reputation scoring was conducted.

Acceptance testing played a key role in the development process of the system. Prototypes were built and presented to the user which they evaluated and the feedback from the users was then incorporated into the next iteration of the system. Often a different selection was used to provide an unbiased feedback as existing users would be familiar with the system after previous tests. However, on the final acceptance testing of the complete system, a mix of new and existing users were requested to test the system to see what new users thought about the final product as well as how existing users thought the system had improved.

\subsection{Testimonials}
In this section we look at some of the testimonials received from users. Testimonials were important to collect as they not only give attestation to the success of the system, but also offer up constructive criticisms that will be useful for further system development.

\begin{displayquote}
	\enquote{It's really interesting how Fidelis is aiming to tackle the issues which exist in social networking. It is definitely a site I would be interested in using, although it will be difficult for it to replace the big names such as Facebook and Instagram in terms of marketing.}
	
	- Elenya McCue, \textit{Engineering BEng, University of Warwick}
\end{displayquote}

\begin{displayquote}
	\enquote{Fidelis is the whole package. A clever idea, backed up with a clean and intuitive design. Aside from the stark similarity to Twitter this idea can definitely go places.}
	
	- Joshua Greenwood, \textit{Advanced Computer Science MPhil, University of Cambridge}
\end{displayquote}

\begin{displayquote}
	\enquote{The design is sleek, clean and familiar. The more I use it, the more posts interest me. It's like it knows me} 
	
	- Luke Vincent, \textit{Graduate Technology Analyst, Barclays PLC}
\end{displayquote}

\begin{displayquote}
	\enquote{Fidelis seems like a different kind of social network. The features it has would allow people to have better interactions compared to other networks like Facebook or Twitter and stop anyone from seeing anything they wouldn't want to. I guess the next step is how you guys plan on making money from this?}
	
	- Hannah McAloone, \textit{Chemistry MSC, University of Warwick}
\end{displayquote}

\noindent The testimonials above, collected from a diverse group of individuals highlight the potential Fidelis has as a product that can disrupt and innovate in the current social network market. 