\chapter{Introduction}
\label{Chapter:Introduction}

Social media has been a rapidly growing industry in the 21st century, with the social network Facebook worth just under \$315 billion dollars as of May 2016 despite only being founded in 2004 \cite{Forbes:Facebook}. However, with the popularity of Facebook and other social networks such as Twitter and Instagram, numerous social issues have arisen which are yet to be fully addressed. Fidelis is an alternative to these networks, which attempts to address these issues. This specification discusses the requirements of the social network Fidelis, as well as the strategies which will be used to design, develop and then test the system.

\section{Problem Statement}
\label{Section:ProblemStatement}

\section{Project Aims}
Abuse detection, provision of user-specific content and reputation scoring are three key features which can be implemented to ensure that a social network is trustworthy. Therefore, the aim of this project is to build a new social network called Fidelis, which will demonstrate the methods of implementing these features and to evaluate how effective these methods are in solving the issues which established social networks currently face in their attempt to earn and retain the trust of their users.

\section{Project Motivation}
As mentioned in \ref{Section:ProblemStatement}, the popular social networks of today have given rise to numerous social issues. One of these such issues is `trolling', with `The Guardian' reporting that ``one in four teenagers suffered hate incidents online last year'' \cite{Gani:Trolling}. In addition, there is an issue with the `echo chamber' effect, where ``users of the social media site interact most with those who share their political views'' \~cite{Jackson:EchoChamber}. This causes a distortion between the ideas perceived on social media and opinions in the real world.

The social networks' current failure to find an effective solution to these pitfalls of social media therefore offers the question of whether a network could be created, which tailors the content users see based on what that particular user would prefer to see. This does not only include filtering any abusive posts, but also offering them diverse content which is of interest to them, either based on the theme of the content or who posted it, whilst limiting the echo chamber effect.

\section{Project Stakeholders}
The internal stakeholders for this project are the development team - Isheanesu Gambe, Naqash Tanzeel, Thomas Mcaloone and Jordan Olney - and Dr Matthew Leeke, who is the project supervisor. To ensure the project is successful, the wider public will also be involved to offer feedback on the progress of Fidelis throughout the duration of the project. In order to test the system, data from other social networks will be required to populate the database with realistic posts. Therefore, the Twitter accounts of multiple stakeholders, both external and internal, will be used, so that large quantities of data can be collected in a short period of time, without exceeding the rate limits imposed by the Twitter API.

\section{Report Structure}
The purpose of this report is to provide a comprehensive account of the process undertaken whilst developing the system associated with the project. The report has been broke down into 3 main sections which identify the high level stages this project went through.

\paragraph{Research and Analysis}
An introduction to the problem being faced and combatted, motivations behind the undertaking of this project and an analysis of any stakeholders is presented in chapter \ref{Chapter:Introduction}. Chapter \ref{Chapter:Research} discusses and analyses any existing solutions, along with any technologies that may be used throughout the project, listing their advantages. Chapter \ref{Chapter:Issues} briefly discusses any issues that may arise and must be considered, whilst chapter \ref{Chapter:SystemRequirements} outlines the original and final requirements for the system.

\paragraph{Development and Testing}
Chapter \ref{Chapter:Design} discusses the thought process behind the designing of the system, whilst chapter \ref{Chapter:Implementation} details the implementation of the system. Chapter \ref{Chapter:Testing} outlines the testing procedures that were carried out, prior to, throughout, and post development. Collectively chapters \ref{Chapter:Design}-\ref{Chapter:Testing} cover the development process from start to finish.

\paragraph{Evaluation and Reflection}
The final 3 chapters reflect on the entire process of developing the system. Chapter \ref{Chapter:ProjectManagement} discusses the project management strategies employed to tackle the project whereas chapter \ref{Chapter:Evaluation} provides an analysis of the work carried out and how well it satisfies the initial requirements of the project. Finally, chapter \ref{Chapter:Conclusion} concludes with a summary and any suggestions for extending the system further.
