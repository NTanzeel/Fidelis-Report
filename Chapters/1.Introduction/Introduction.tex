\chapter{Introduction}
\label{Chapter:Introduction}

Social networking platforms are a well established part of day-to-day life for many people. Figure \ref{fig:SocialMediaRegionGender} shows how users spend multiple hours on social media every month. With such a large number of users spending large amounts of time using these platforms, it is clear that they have a significant influence on modern life and this ability to attract users has earned social media companies significant amounts of money. For example, Facebook, which is one of the most prominent social media sites, is worth just under \$315 billion dollars as of May 2016 despite only being founded in 2004 \cite{Forbes:Facebook}. However, with the popularity of Facebook and other social networks such as Twitter and Instagram, numerous social issues have arisen which are yet to be fully addressed. Fidelis is an alternative to these networks, which attempts to address these issues. This report provides an overview of the requirements of the social network Fidelis, as well as the strategies used to design, develop and then test the system and a review of what this project has achieved.

\begin{figure}[H]
  \centering
  \includegraphics[width=0.75\textwidth]{Images/Introduction/SocialMediaRegionGender}
  \caption{User engagement June 2015, by region and gender \cite{Statista:SocialMediaRegionGender}} \label{fig:SocialMediaRegionGender} 
\end{figure}


\section{Problem Statement}
\label{Section:ProblemStatement}
Despite the successfulness of social networks, their ability to solve issues with regards to the trustworthiness of their content has often been criticised. For example, Figure \ref{fig:TrollingByTopic} shows a number of topics on which respondents had witnessed internet `trolling' in a 2014 survey, which is where users ``make a deliberately offensive or provocative online post with the aim of upsetting someone or eliciting an angry response from them'' \cite{Oxford:Trolling}. This data demonstrates the prevalence of online abuse, with 65\% of those questioned responding that they have witnessed it in some form. Furthermore, ``one in four teenagers suffered hate incidents online last year'' \cite{Gani:Trolling}.

\begin{figure}[H]
  \centering
  \includegraphics[width=0.75\textwidth]{Images/Introduction/TrollingByTopic}
  \caption{Internet trolling: topics online in the U.S. in 2014 \cite{Statista:TrollingByTopic}} \label{fig:TrollingByTopic} 
\end{figure}

In addition to online abuse, the echo chamber effect is another issue in social networking which is becoming increasingly prevalent, whereby ``users of the social media site interact most with those who share their views'' \cite{Jackson:EchoChamber}. Facebook in particular have been accused of not doing enough to prevent this echo chamber effect, with former Times editor and News International chief executive Robert Thomson allegedly stating that Facebook ``not only help to promote `fake news', but will also reduce the diversity of opinion'' as a result of them ``routinely and selectively `unpublishing' certain views and news'' \cite{Orlowski:EchoChamber}. This causes a distortion between the ideas perceived on social media and opinions in the real world.

The inability for social networks to effectively address these issues has caused some users to lose trust in the current social networks, with Twitter struggling to attract new users \cite{Barrons:Twitter}. If this trend continues or worsens, the success which social networks have had could decrease.

\section{Project Aims}
Social networks' current failure to find an effective solution to these pitfalls therefore offers the question, ``Can a network be created, that tailors the content users see based on what they would prefer to see?''. This does not only include protecting the users by filtering any abusive posts, but also offering them diverse content that is of interest to them, either based on the theme of the content or who posted it. By providing content based on the topics a user is interested in, even if that content offers an opposing viewpoint to the user's, the echo chamber effects should also be reduced, which will help create a more open discussion amongst users. The aim of this project is to build a new social network, Fidelis, which will demonstrate the methods of implementing these features and to evaluate how effective these methods are in solving the issues which established social networks currently face in their attempt to earn and retain the trust of their users.

\section{Project Motivation}
Current social networks failure to find solutions to the social problems which arise on their platforms has created an opportunity for a new social network to be built, which can effectively handle these issues. With Twitter struggling to attract new users \cite{Barrons:Twitter}, potential users may be drawn towards Fidelis if it can provide a platform for open debate on topics which interest the user, without the risk of receiving abuse. In fact, since beginning this project, Twitter have made further efforts to reduce the amount of abuse which appears on their platform. This includes identifying accounts as they are engaging in abusive behaviour, even if this behaviour hasn't been reported and collapsing potentially abusive or low-quality Tweets \cite{Twitter:Safety}, both of which are features being implemented as a part of this project. With Twitter taking steps to make their site safer by implementing similar features to Fidelis highlights the relevance of such a project at this current time.

The success of this project could play a role in the future success of social networks. As proved by the wealth of current social networks, it is of great importance to social networks to attract and retain active users. This is because popular social networks are able to attract advertisers who are willing to pay for their adverts to be displayed on the site \cite{Investopedia:Adverts}. The large amounts of data which is collected as a result of users interacting on the platform can also be valuable, because ``data can be used to optimize the way it does business: acquisition, retention, targeting, pricing, etc.'' \cite{TechCrunch:Data}.

An additional motivation, other than financial, is the impact `trolling' is having on society. There have been multiple cases of suicides as a result of cyber-bullying on social networks \cite{NoBullying:Cyber}. With effective abuse detection and prevention, the statistics around cyber-bullying amongst users could improve.

\section{Project Stakeholders}
The internal stakeholders for this project are the development team - Isheanesu Gambe, Naqash Tanzeel, Thomas Mcaloone and Jordan Olney - and Dr Matthew Leeke, who is the project supervisor. To ensure the project is successful, the wider public will also be involved to offer feedback on the progress of Fidelis throughout the duration of the project. In order to avoid a cold start, data from other social networks will be required to provide Fidelis with realistic posts, which can be used to train machine learning models. Therefore, the Twitter accounts of multiple stakeholders, both external and internal, will be used, so that large quantities of data can be collected in a short period of time, without exceeding the rate limits imposed by the Twitter API.

\section{Report Structure}
The purpose of this report is to provide a comprehensive account of the process undertaken whilst developing the system associated with the project. The report can been broken down into 3 main sections which identify the high level stages this project went through, which are:

\paragraph{Research and Analysis}
An introduction to the problem being faced and combated, motivations behind the undertaking of this project and an analysis of any stakeholders is presented in Chapter \ref{Chapter:Introduction}. Chapter \ref{Chapter:Research} discusses and analyses existing solutions, along with any technologies that may be used throughout the project, listing their advantages. Chapter \ref{Chapter:Issues} briefly discusses any issues that may arise and must be considered, whilst Chapter \ref{Chapter:SystemRequirements} outlines the original and final requirements for the system.

\paragraph{Development and Testing}
Chapter \ref{Chapter:Design} discusses the thought process behind the designing of the system, whilst chapter \ref{Chapter:Implementation} details the implementation of the system. Chapter \ref{Chapter:Testing} outlines the testing procedures that were carried out, prior to, throughout, and post development. Collectively chapters \ref{Chapter:Design}-\ref{Chapter:Testing} cover the development process from start to finish.

\paragraph{Evaluation and Reflection}
The final 3 chapters reflect on the entire process of developing the system. Chapter \ref{Chapter:ProjectManagement} discusses project management strategies employed to tackle the project whereas Chapter \ref{Chapter:Evaluation} provides an analysis of the work carried out and how well it satisfies the initial requirements of the project. Finally, Chapter \ref{Chapter:Conclusion} concludes with a summary and any suggestions for extending the system further.
