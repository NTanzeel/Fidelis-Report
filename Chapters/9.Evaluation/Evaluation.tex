\chapter{Evaluation}
\label{Chapter:Evaluation}
After a comprehensive look at system design, implementation and testing, focus now turns to performing an evaluation of how well the completed system aligns with the original, envisioned idea. To this end, system requirements will be evaluated to assess how closely they met the requirements set out in Chapter \ref{Chapter:SystemRequirements}. Assessment will occur as a pass-fail test, coupled with a brief comment on each requirement. Additionally an evaluation of the legal, social, ethical and professional issues affecting the project, discussed in Chapter \ref{Chapter:Issues}, will also be provided.

\section{Requirements}
This section of the report will look at both functional and non-functional requirements. Evaluation of each requirement, as set out in Chapter  \ref{Chapter:SystemRequirements}, is done using a pass-fail test. Along with tables detailing evaluation results for functional and non-functional requirements, the following sections will also provide justification for any requirements that have not been met.

\subsection{Functional Requirements}
Table \ref{tab:functional-eval} shows the results from functional requirements evaluation. This is essential in identifying whether the product has met its goals and provides the functionality set out by the project stakeholders. All functional requirements set in the project specification, and later refined, were met; the system adheres to and is capable of performing all of the desired functionalities. 

\begin{longtabu} to \textwidth {XXXX}
\hline
Requirement & Description & Comment & Result \\ 
\hline
\textbf{F1} & The system must be able to communicate with a number of third party APIs in order to retrieve data & The Twitter API is used to retrieve tweets that are used to train the classifier for automatic post categorisation \vspace{2mm} & \textcolor{passgreen}{PASS} \\
\textbf{F2} & Users must be able to register and log into the system & The system allows for users to register as a new user or log in as an existing user \vspace{2mm} & \textcolor{passgreen}{PASS} \\
\textbf{F3} & A user may make their account private, preventing other users from being able to see the content they share & Users have the option to toggle privacy on their accounts. All accounts are public by default \vspace{2mm} & \textcolor{passgreen}{PASS} \\
\textbf{F4} & A user may block another user & Users have the option to block other users using a `block' button \vspace{2mm} & \textcolor{passgreen}{PASS} \\
\textbf{F5} & In order to see content, users should be able to add other users to their ``trust circle'' \vspace{2mm} & Users can add to their ``trust circle'' by following other users & \textcolor{passgreen}{PASS} \\
\textbf{F6} & The system will overcome the cold-start problem, under which a user will not see anything on their timeline upon registration, by providing a set of predefined categories which can be used to explore &                                                                                            Fidelis has the following default categories, available to the user as soon as they register: Education, Fashion, Finance, Fine Arts, Food, Health, Home, Miscellaneous, Politics, Politics, Sport and Travel technology \vspace{2mm} & \textcolor{passgreen}{PASS}  \\
\textbf{F7} & The system will provide a personal feed where the user is able to view content from the people they trust, prioritised by the reputation of the people they follow as well as the reputation of the content itself  \vspace{2mm} & Each user has a feed for the users they follow. This feed contains content from these users which is ranked by reputation of the author and their content & \textcolor{passgreen}{PASS} \\
\textbf{F8} & A user must be able to post new content to the system                                                                                                                                                              &   Users are able to post new content on the homepage. When submitted, this post is appended to the users' personal feed. Within the post, the user can mention specific individuals by using their username preceded by an @ symbol \vspace{2mm} & \textcolor{passgreen}{PASS} \\
\textbf{F9} & Users can view any posts made by public accounts or by private accounts which they follow & Users can only view posts made by a public account. To be able to view content from private accounts, users must first make a request to follow the private account \vspace{2mm} & \textcolor{passgreen}{PASS} \\
\textbf{F10} & Users must be able to reply to a post made by another user &  Users can reply to a post in a similar manner to when posting new content. When a comment is made, the author of the post is notified of this. Replying to a post starts a thread of comments akin to Facebook commenting. Within comments, users can mention specific users. This also notifies the relevant user \vspace{2mm} & \textcolor{passgreen}{PASS} \\
\textbf{F11} & Users may interact with content they come across by liking or disliking it. Each of these actions will impact the reputation of the content \vspace{2mm} & The user can like and dislike posts by using the thumbs up and down icons which accompany every post & \textcolor{passgreen}{PASS} \\
\textbf{F12} & Users are notified of any collaboration events involving their account & The user receives a notification whenever another user requests to follow them, or whenever their content is liked, disliked or commented on \vspace{2mm} & \textcolor{passgreen}{PASS} \\
\textbf{F13} & Users may change the appearance of their post & The user is able to upload their own personalised profile picture and cover picture to appear on their profile and throughout Fidelis \vspace{2mm} & \textcolor{passgreen}{PASS} \\
\textbf{F14} & The system will automatically calculate a reputation score for users which symbolises the trustworthiness of the user and the content they post & Using an abuse detection classifier, along with a script that calculates post reputation, user reputation will be determined by the content they post on Fidelis \vspace{2mm} & \textcolor{passgreen}{PASS} \\
\textbf{F15} & The system will also calculate a reputation for each user posted content to represent the quality of content & A Python script, which is run periodically, will retrieve user posts and calculate the reputation for each post based on the votes on it. Additionally, if a post is deemed abusive its reputation will reflect this \vspace{2mm} & \textcolor{passgreen}{PASS} \\
\textbf{F16} & Each user will be able to choose how the notion of trust is implemented & Users are able to toggle between multiple representations of trust \vspace{2mm} & \textcolor{passgreen}{PASS} \\
\textbf{F17} & The system will provide recommendations to the user & User and content recommendations are generated for users. User recommendations are available through a widget across the home and profile pages, and content recommendations are available on the discover page \vspace{2mm} & \textcolor{passgreen}{PASS} \\
\textbf{F18} & The system will allow the user to choose how recommendations are generated for them & Users are able to toggle settings related to how recommendations are generated for them. Namely these are the reputation of users and content, the number of recommendations to generate and a similarity threshold \vspace{2mm} & \textcolor{passgreen}{PASS} \\
\textbf{F19} & The system will incorporate abuse detection to prevent things such as profanity, swearing and nudity amongst other things & Fidelis has in place an abuse detection classifier that is able to detect abusive content. Similarly to the script calculating reputations, the classifier will be run periodically on user posts. Users are also able to manually report content they find offensive \vspace{2mm} & \textcolor{passgreen}{PASS} \\
\textbf{F20} & Users can report content which they find is inappropriate or offensive, such as profanity or nudity                                                                                                                &  Users can flag posts they find inappropriate using the flag icon that accompanies each post. Flagged items are used to re-train the abuse detection classifier to increase its accuracy in detecting abusive content                                                                                             \vspace{2mm} & \textcolor{passgreen}{PASS} \\ 
\hline
%\end{tabular}
\caption{Evaluation of Functional Requirements}
\label{tab:functional-eval}
\end{longtabu}

\subsection{Non-Functional Requirements}
Non-functional requirements ensure that the developed solution is applicable to usage in the real word. Functional requirements may ensure that the system provides all the features required, but these could be entirely useless if the system is not user friendly. Table \ref{tab:nonfunctional-eval} discusses all the non-functional requirements and how closely these requirements were met. The comments column discusses how the requirement was met and whether any aspects of the requirement were missed. As visible in the table, all requirements were passed; the system goes beyond just being functional and additionally meets, amongst other things, criteria set for usability, scalability and extensibility.

\begin{longtabu} to \textwidth {XXXX}
\hline
Requirement & Description & Comment & Result \\ 
\hline
\textbf{NF1} & Compatibility & Fidelis is compatible across all devices. Using Bootstrap for front-end development, cross-platform compatibility was guaranteed. System functionality is compatible for mobile devices, but some features may be hidden \vspace{2mm} & \textcolor{passgreen}{PASS} \\
\textbf{NF2} & Usability & The decision to imitate Twitter UI design meant that the system inherited designs that have been refined over time to provide a good user experience \vspace{2mm} & \textcolor{passgreen}{PASS} \\
\textbf{NF3} & Security & All data held by Fidelis is stored in a password-protected database. Additionally, user credentials are encrypted when stored. The option to make user accounts private add an additional security layer to protecting user data \vspace{2mm} & \textcolor{passgreen}{PASS} \\
\textbf{NF4} & Scalability & Whilst all decisions taken have been taken with storage and processing power in mind, larger datasets may result in slow response from the system on the current hardware. Due to the fact that users are allowed to upload resources, it is difficult to contain storage within a limit and hardware upgrades will be necessary as the system grows. It is also impossible to test this requirement without a large user base \vspace{2mm} & \textcolor{passgreen}{PASS} \\
\textbf{NF5} & Extensibility                                                                                                                         & The system was developed using a modular approach made possible through the use of an MVC approach. This means that the system can be expanded at any time as each module is completely independent of another and can be upgraded or replaced with ease. This also allows new functionality to be integrated into the system with ease \vspace{2mm} & \textcolor{passgreen}{PASS} \\
\textbf{NF6} & Maintainability & The system is made maintainable by the documentation on all of the PHP code written throughout the project. Good coding practices as laid out by the frameworks and PHP manual were used. In addition, version control was used with through details provided for each commit. A regular update was performed for the Laravel framework to ensure that the latest version was being used. All third party apis use a CDN which means they're updated as soon as new version become available \vspace{2mm} & \textcolor{passgreen}{PASS} \\
\textbf{NF7} & Readability & As mentioned in maintainability, the code has been well documented. This is done only for the PHP code as the remaining code is relatively self explanatory. All changes were logged through version control as well as other progress tracking tools and a clear outline is provided in this report \vspace{2mm} & \textcolor{passgreen}{PASS} \\
\hline
\caption{Evaluation of Non-Functional Requirements}
\label{tab:nonfunctional-eval}
\end{longtabu}

\section{Legal, Social, Ethical and Professional Issues}
At the outset of this project, a number of legal, ethical, social and professional issues that could affect the Fidelis platform were identified. These issues were discussed fully in Chapter \ref{Chapter:Issues}. It is critical to review the issues discussed and identify how they have been addressed, and as such this section will look at these issues were confronted. In terms of legal issues, there could be concern that the content/activity of a user may have implications for Fidelis, for example a user posting copyright infringing content. It has therefore been outlined in the Fidelis Terms of Service that a user is responsible for their own content and activity on the Fidelis platform and also that any content deemed illegal may be removed. In order to comply with copyright law during development of the platform, it was ensured that each team member was made aware of the laws outlining which content could be used. The platform was developed largely using Laravel, an open-source php framework \cite{Laravel:Home}. The Fidelis logo is an original and so can be used with no permissions needed and other images and icons around the website were collected from open-source font libraries such as font-awesome \cite{FontAwesome}.

Social issues included the consideration of social networks and their relationship with mental health. It has been made clear that users should only post content they are comfortable with sharing to avoid users regretting doing so and the support section of the website suggests some ways a user can find help. This includes contact details and website of Samaritan's \cite{Samaritans:Home}, a charity providing assistance in talking over any issues a person may be having in their life and offering support to help that person overcome these issues. Additionally, the user may wish to change their settings for abuse detection. Our abuse detection system is designed to identify any content that the user may find abusive or offensive in any way and remove it. This ensures they only see content they wish to see. Unfortunately, this system cannot be relied upon fully, as occasionally there may be some content that an individual may take issue with which is not identified by the system as being abusive. In these cases, users can make use of the ability to report content. This will prevent them from seeing this content again and will affect the reputation of the content and the user that posted it, making it less likely that others will see the same content.

There are also ethical considerations for the Fidelis platform. It may be the case that users may not wish for everybody to see their posts. It has therefore been ensured that users can set their accounts to `private' on their settings page. This will prevent non-approved users from seeing the content they share. In addition there is the potential for the bias of any person(s) involved in the development of the Fidelis platform to affect the content recommended. Fidelis has been designed and implemented in such a way that content recommendation and filtering is based on user preferences as well as the content the user interacts with and they should primarily be shown content from other users that post `good' content, i.e., content/users with a high reputation. There is no reason that the views of anyone involved in developing the platform should affect the end-user content.

On a professional level it was noted that the Fidelis platform is required to adhere to the Data Protection of 1998 \cite{DPA}. In order to comply, it has been ensured that the database is as secure as possible. Sensitive data such as user passwords are encrypted to protect their information and any request to interact with data from the database (both retrieving and inserting data) requires checks such as ensuring the user is logged into an account and cannot interact with data related to another user. This way no individual can have access to data that they are not authorised to access, maintaining data confidentiality.

\section{Project Management}
The decision to use agile techniques for development, as discussed in Chapter \ref{Chapter:ProjectManagement}, enabled the project to evolve in parallel with changing project requirements. Using these techniques also provided developers with a creative freedom, which in itself played a hand in refining requirements. Adopting a ``custom'' agile approach that facilitated extensive documentation was a good idea. This addition to agile techniques introduced structure to the project. Changes made to the project timeline occurred, and are discussed again in Chapter \ref{Chapter:ProjectManagement}, but as mentioned these did not affect development work and were dealt with effectively due to chosen methodologies. The tools employed for both development and management were a success. Development tools were current and up to date, meaning all development work done will age well. The management tools used meant that all work was done transparently, and created multiple channels through which work could be discussed and evaluated. The risk analysis performed at the start of the project set it up to be able to cope with any minor or major risk. Fortunately, no major risks were realised. Few developer illnesses and unavailability plagued the project, but these events were planned for and handled effectively.