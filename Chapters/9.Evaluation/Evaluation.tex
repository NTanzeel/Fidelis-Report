\chapter{Evaluation}
\label{Chapter:Evaluation}
After a comprehensive look at system design, implementation and testing, focus now turns to performing an evaluation of how well the completed system aligns with the original, envisioned idea. To this end, system requirements will be evaluated to assess how closely they met the requirements set out in Chapter \ref{Chapter:SystemRequirements}. Assessment will occur as a pass-fail test, coupled with a brief comment on each requirement. Additionally an evaluation of the legal, social, ethical and professional issues affecting the project, discussed in Chapter \ref{Chapter:Issues}, will also be provided.

\section{Requirements}
This section of the report will look at both functional and non-functional requirements. Evaluation of each requirement, as set out in Chapter  \ref{Chapter:SystemRequirements}, is done using a pass-fail test. Along with tables detailing evaluation results for functional and non-functional requirements, the following sections will also provide justification for any requirements that have not been met.

\subsection{Functional Requirements}
Table \ref{tab:functional-eval} shows the results from functional requirements evaluation. This is essential in identifying whether the product has met its goals and provides the functionality set out by the project stakeholders. All functional requirements set in the project specification, and later refined, were met; the system adheres to and is capable of performing all of the desired functionalities. 

\begin{longtabu} to \textwidth {XXXX}
\hline
Requirement & Description                                                                                                                                                                                                        & Comment                                                                                       & Result                      \\ \hline
\textbf{F1}          & The system must be able to communicate with a number of third party APIs in order to retrieve data                                                                                                                 & The Twitter API is used to retrieve tweets that are used to train the classifier for automatic post categorisation                                                           & {\color[HTML]{34FF34} PASS} \\
\textbf{F2}          & Users must be able to register and log into the system                                                                                                                                                             & The system allows for users to register as a new user or log in as an existing user           & {\color[HTML]{34FF34} PASS} \\
\textbf{F3}          & A user may make their account private, preventing other users from being able to see the content they share                                                                                                        & Users have the option to toggle privacy on their accounts. All accounts are public by default & {\color[HTML]{34FF34} PASS} \\
\textbf{F4}          & A user may block another user                                                                                                                                                                                      & Users have the option to block other users using a 'block' button                             & {\color[HTML]{34FF34} PASS} \\
\textbf{F5}          & In order to see content, users should be able to add other users to their ``trust circle''                                                                                                                         & Users can add to their ``trust circle'' by following other users                              & {\color[HTML]{34FF34} PASS} \\
\textbf{F6}          & The system will overcome the cold-start problem, under which a user will not see anything on their timeline upon registration, by providing a set of predefined categories which can be used to explore            &                                                                                            Fidelis has the following default categories, available to the user as soon as they register: Education, Fashion, Finance, Fine Arts, Food, Health, Home, Miscellaneous, Politics, Politics, Sport and Travel
technology &  {\color[HTML]{34FF34} PASS}  \\
\textbf{F7}          & The system will provide a personal feed where the user is able to view content from the people they trust, prioritised by the reputation of the people they follow as well as the reputation of the content itself &                                                                                              Each user has a feed for the users they follow. This feed contains content from these users which is ranked by reputation of the user and their content & {\color[HTML]{34FF34} PASS} \\
\textbf{F8}          & A user must be able to post new content to the system                                                                                                                                                              &   Users are able to post new content on the homepage. When submittied, this post is appended to the users' personal feed . Within the post, the user can mention specific individuals by using their username preceded by an \@ symbol                                                                                            & {\color[HTML]{34FF34} PASS} \\
\textbf{F9}          & Users can view any posts made by public accounts or by private accounts which they follow                                                                                                     &      Users can only view posts made by a public account. To be able to view content from private accounts, users must first make a request to follow the private account  & {\color[HTML]{34FF34} PASS} \\
\textbf{F10}         & Users must be able to reply to a post made by another user                                                                                                                                                         &  Users can reply to a post in a similar manner to when posting new content. When a comment is made, the author of the post is notified of this. Replying to a post starts a thread of comments akin to Facebook commenting. Within comments, users can mention specific users. This also notifies the relevant user                                                                                             & {\color[HTML]{34FF34} PASS} \\
\textbf{F11}         & Users may interact with content they come across by liking or disliking it. Each of these actions will impact the reputation of the content                                                               &                                                                                              The user can like and dislike post by using the thumbs up and down icons which accompany every post. It is only possible to vote on a post, and not comments. & {\color[HTML]{34FF34} PASS} \\
\textbf{F12}         & The system will automatically calculate a reputation score for users which symbolises the trustworthiness of the user and the content they post                                                                    &                                                                                              Using an abuse detection classifier, along with a script that calculates post reputation, user reputation will be determined by the content they post on Fidelis  & {\color[HTML]{34FF34} PASS} \\
\textbf{F13}         & The system will also calculate a reputation for each user posted content to represent the quality of content                                                                                                       & A Python script will retrieve user posts and calculate the reputation for each post based on the votes on it. This script will be run periodically, meaning reputations may not always be accurate. Additionally, if a post is deemed abusive its reputation will reflect this & {\color[HTML]{34FF34} PASS} \\
\textbf{F14}         & Each user will be able to choose how the notion of trust is implemented                                                                                                      &    Users are able to toggle...                                                                                            & {\color[HTML]{34FF34} PASS} \\
\textbf{F15}         & The system will provide recommendations to the user &    User and content recommendations are generated for users. User recommendations are available through a widget across the home and profile pages, and content recommendations are available on the discover page & {\color[HTML]{34FF34} PASS} \\
\textbf{F16}         & The system will allow the user to choose how recommendations are generated for them   &    Users are able to toggle settings related to how recommendations are generated for them. Namely these are the reputation of users and content, the number of recommendations to generate and a similarity threshold & {\color[HTML]{34FF34} PASS} \\
\textbf{F17}         & The system will incorporate abuse detection to prevent things such as profanity, swearing and nudity amongst other things                                                                                          & Fidelis has in place an abuse detection classifier that is able to detect abusive content. Similarly to the script calculating reputations, the classifier will be run periodically on user posts                                                                                                & {\color[HTML]{34FF34} PASS} \\
\textbf{F18}         & Users can report content which they find is inappropriate or offensive, such as profanity or nudity                                                                                                                &  Users can flag posts they find inappropriate using the flag icon that accompanies each post. Flagged items are used to re-train the abuse detection classifier to increase its accuracy in detecting abusive content                                                                                             & {\color[HTML]{34FF34} PASS} \\ \hline
%\end{tabular}
\caption{Evaluation of Functional Requirements}
\label{tab:functional-eval}
\end{longtabu}

\subsection{Non-Functional Requirements}
Non-functional requirements ensure that the developed solution is applicable to usage in the real word. Functional requirements may ensure that the system provides all the features required, but these could be entirely useless if the system is not user friendly. Table \ref{tab:nonfunctional-eval} discusses all the non-functional requirements and how closely these requirements were met. The comments column discusses how the requirement was met and whether any aspects of the requirement were missed. 

As visible in the table, the only requirement that was not fully satisfied and hence didn't receive a pass was requirement \textbf{NF4} regarding scalability. This is because although all the decisions were made with storage and other resources in mind, some of the these were made due to the nature of the service and could not be compromised to meet this requirement. This is especially true for the uploading of resources, which is a necessary part of the system but uses up a significant amount of storage space. This was optimised as far as possible but at a point the hardware will not be sufficient and will have to be expanded. In addition to this, as the system grows, having one server will not be sufficient enough to deal with the load and the network will need to be expanded.

\begin{longtabu} to \textwidth {XXXX}
\hline
Requirement & Description                                                                                                                                                                                                        & Comment                                                                                       & Result                      \\ \hline
\textbf{NF1}          & Compatibility      & Fidelis is compatible across all devices. Using Bootstrap for front-end development, cross-platform compatibility was guaranteed. System functionality is compatible for mobile devices, but some features may be hidden  & {\color[HTML]{34FF34} PASS} \\
\textbf{NF2}          & Usability &       The decision to imitate Twitter UI design meant that the system inherited designs that have been refined over time to provide a good user experience     & {\color[HTML]{34FF34} PASS} \\
\textbf{NF3}          & Security & All data held by Fidelis is stored in a password-protected database. Additionally, user credentials are encrypted when stored. The option to make user accounts private add an additional security layer to protecting user data & {\color[HTML]{34FF34} PASS} \\
\textbf{NF4}          & Scalability &  Whilst all decisions taken have been taken with storage and processing power in mind, larger datasets may result in slow response from the system on the current hardware. Due to the fact that users are allowed to upload resources, it is difficult to contain storage within a limit and hardware upgrades will be necessary as the system grows. It is also impossible to test this requirement without a large user base & {-} \\
\textbf{NF5}          & Extensibility                                                                                                                         & The system was developed using a modular approach made possible through the use of an MVC approach. This means that the system can be expanded at any time as each module is completely independent of another and can be upgraded or replaced with ease. This also allows new functionality to be integrated into the system with ease.      & {\color[HTML]{34FF34} PASS} \\
\textbf{NF6}          & Maintainability   &                                                                                         The system is made maintainable by the documentation on all of the PHP code written throughout the project. Good coding practices as laid out by the frameworks and PHP manual were used. In addition, version control was used with through details provided for each commit. A regular update was performed for the Laravel framework to ensure that the latest version was being used. All third party apis use a CDN which means they're updated as soon as new version become available. & {\color[HTML]{34FF34} PASS}\\
\textbf{NF7}          & Readability &                                                                                      As mentioned in maintainability, the code has been well documented. This is done only for the PHP code as the remaining code is relatively self explanatory. All changes were logged through version control as well as other progress tracking tools and a clear outline is provided in this report. & {\color[HTML]{34FF34} PASS} \\\hline
\caption{Evaluation of Non-Functional Requirements}
\label{tab:nonfunctional-eval}
\end{longtabu}

\section{Legal, Social, Ethical and Professional Issues}
At the outset of this project, a number of legal, ethical, social and professional issues that could affect the Fidelis platform were identified. It is critical to review these issues and identify how they have been addressed, and as such this section will look at these issues were confronted.

\subsection{Legal Issues}
Since the proposed system focuses on users creating and sharing content, it is difficult to control what users may be posting. The platform is intended to promote an environment for free speech and sharing ideas/opinions, however there may be cases where a user publicly shares some information which was intended to be private or confidential. This could be personal information, or information pertaining to a third party such as an employer. Once information has been shared in this way there is little recourse; as soon as it is public, anyone can view it. In such a case one may argue that since the information was posted and shared through the Fidelis platform, the liability is with the platform and not with the user. It was therefore important to highlight in the Fidelis `Terms of Use' that users are responsible for any content that they post or share through this platform.

Just as third-party materials posted to a social media site may infringe copyright or trademarks, posting photographs and video without proper releases may violate the privacy or publicity rights of individuals. Additionally, within certain industries, employees must ensure that they do not violate specific privacy regulations of their employer in their activities on social media sites. Content such as this can be reported on the Fidelis platform and the right is reserved to remove any content that should not be allowed on the site, such as in the case of a copyright request. Any copyright claim against Fidelis would be investigated and if necessary the offending content would be removed.

In addition to ensuring users respect and uphold copyright and trademarks, it is important that the platform itself does not make use of and unlicensed materials. This includes, but is not limited to, using copyright-free images and ensuring that the required permissions are fulfilled for any software used in the creation and running of the platform. These guidelines were clearly relayed to all project team members to ensure no legal violations occurred.

There is a responsibility to ensure that all users of the Fidelis platform are being treated both fairly and legally. A common trend across many social media platforms is targeted `abuse' towards a specific individual. This can be in the form of `cyber bullying', `trolling' or in some extreme cases, defamation. Defamation is defined as ``A false statement or fact, not made under privilege, that is communicated to a third person and that causes damage to a persons reputation. For public figures, the plaintiff must also prove actual malice.'' [40]. Fidelis has been created around focusing on filtering the content that is presented to a user, which is intended to reduce the amount of negative content the user will see. While a user can choose to hide content they see as negative, the posts will still persist on the platform. Therefore any content considered defamatory is still subject to defamation law. The Fidelis `Terms of Use', the right to remove content that is illegal, spam, or abusive is reserved. This means that any content violating these criteria are subject to being manually deleted. If content is illegal there is also a duty to ensure that the law is followed, and aid authorities as appropriate.

It should therefore be possible to delete these posts if needed as well as possibly using the posts to find the IP address or some other information about the original poster (i.e. to aid authorities whilst following both the law and any policies regarding privacy or anonymity of the user).

\subsection{Social Issues}
Despite one of the main features of Fidelis being abuse detection, it is important to consider the implications should certain abusive posts go undetected. Due to the nature of social media and the difficulty to systematically detect the semantics of a post, it is likely that there will be posts which are not detected by the abuse algorithms. In order to minmise the effects of this fidelis has been designed to allow users to blok other user, report the posts of others and a user has the ability to delete their own content. As well as abusive content, further offensive content may be posted on the site which does not fall under the category of abuse and would therefore not be identified by the abuse detection system within Fidelis. This may include nudity or links to inappropriate or illegal content. Again, the user is be able to remove this content from their feed if they find offensive, but in extreme cases users may have to be reported to the authorities or removed from the Fidelis site altogether.

In addition to abusive content, Fidelis would also be subject to other social issues which have been recognised within preexisting social networks. This includes possible mental health implications of socialising online instead of face-to-face. Research has suggested that ``digital communications less able to lower depression risk'' in comparison with ``people who regularly met in person with family and friends'' [45]. Therefore, it is important that Fidelis attempts to mitigate this risk by encouraging users not to spend too much time on the social network. The Fidelis `Terms of Use' outline that users should only provide content they are comfortable sharing with others. These terms highlight information about how the users should conduct themselves on the platform in order to ensure they are more aware of the outcomes of how they use Fidelis. In additon a number of support options have been provided, advising to report inappropriate behaviour, manage their abuse detection settings and providing contact details for the website and helpline of Samaritan's \cite{Samaritans:Home}, a charity providing assistance in talkin over any issues a person may be having in their life and offering support to help the person overcome these issues.

\subsection{Ethical Issues}
As a social media platform, there is a level of ethical responsibility to protect users by ensuring an agreed upon level of privacy. For example it is fairly common for company HR departments to review the social media pages of both job candidates and current employees. While this practice may be of use to the company, the design of the Fidelis platform is such that users should feel comfortable posting and discussing content without feeling the need to censor themselves. Since users may not wish for everybody to see their posts a user can set their account to `private' on their settings page. This will prevent non-approved users from seeing the content they share.

This system aims to provide a platform for users to post and share content relating to a wide range of topics. In some cases, for example with politics, there are often issues that see people taking very different viewpoints. In order too provide a balanced platform for conversation and debate on these topics, it is crucial to take an unbiased approach to how the system decides which content to show to a user and which users to recommend to each other. It would be very easy for a person designing a system like Fidelis to implement bias based on their own opinions, presenting all users with the same content that only shows one side of an argument. This would be counteractive the maint ideas begind the Fidelis platform and would betray the trust of users. Fidelis has been designed and implemented in such a way that content recommendation and filtering is based on user preferences as well as the content the user interacts with and they should primarily be shown content from other users that post `good' content, i.e., content/users with a high reputation.

\subsection{Professional Issues}
Fidelis requires the storeage large amounts of user data. There is therefore a duty to ensure that the privacy of the users is protected, by abiding by the `Data Protection Act' of 1998. This legislates that data is ``fairly and lawfully processed; processed for limited purposes and not in any manner incompatible with those purposes; adequate, relevant and not excessive; accurate and where necessary, up to date; not kept for longer than is necessary; processed in line with the data subject's rights; secure and that personal information shall not be transferred to countries outside the EEA without adequate Protection'' [18]. In order to comply, it has been ensured that the database is as secure as possible. Sensitive data such as user passwords are encrypted to protect their information and any request to interact with data from the database (both retrieving and inserting data) requires checks such as ensuring the user is logged into an account and cannot retrieve data related to another user.

\section{Project Management}
The decision to use agile techniques for development, as discussed in Chapter \ref{Chapter:ProjectManagement}, enabled the project to evolve in parallel with changing project requirements. Using these techniques also provided developers with a creative freedom, which in itself played a hand in refining requirements. Adopting a ``custom'' agile approach that facilitated extensive documentation was a good idea. This addition to agile techniques introduced structure to the project. Changes made to the project timeline occurred, and are discussed again in Chapter \ref{Chapter:ProjectManagement}, but as mentioned these did not affect development work and were dealt with effectively due to chosen methodologies. The tools employed for both both development and management were a success. Development tools were current and up to date, meaning all development work done will age well. The management tools used meant that all work was done transparently, and created multiple channels through which work could be discussed and evaluated. The risk analysis performed at the start of the project set it up to be able to cope with any minor or major risk. Fortunately, no major risks were realised. Few developer illnesses and unavailability plagued the project, but these events were planned for and handled effectively.