\section{Database}
The successful operation of the system, proposed in figure \ref{fig:SystemArchitecture}, is entirely dependent on the database, which stores and provides most, if not all, of the content for the application. For a system so dependent on its data, it is crucial that care and appropriate measures are taken when designing the database. In order for the application to be efficient, both in terms of speed and storage, it is essential that the database is designed in a way that minimises storage space by finding an optimal balance of normalisation and redundancy. To ensure data consistency, various degrees of normalisation were used when designing and developing the database. Taking posts as an example, one could store the username and other details of the author along with every post to improve page loading speeds but this would lead to unnecessary redundancy and a significant increase in storage space. As a result of this, the data was split across multiple tables, based on attributes, which relied on each other through the use of functional dependencies and relationship constraints to ensure consistency and provide a complete dataset. An overview of the database is provided in section \ref{SubSection:Database_Schema} along with a breakdown of the tables based on components of the system in section \ref{SubSection:Database_Tables}. An explanation of the normalisation techniques, dependencies and security are also provided in sections \ref{SubSection:Database_Normalisation}, \ref{SubSection:Database_Constraints}, and \ref{SubSection:Database_Security} respectively.

\subsection{Schema}
\label{SubSection:Database_Schema}
In line with the agile methodology and component driven design approach adopted for the development of the project, the database was also designed with modularity in mind. This allows the database to be expanded as new system components were developed. Each component stores its data in independent tables which rely on previously implemented tables through relational dependencies. The diagram in figure \ref{fig:Database_ERD} shows the set of tables in the database, and the relationship between these tables, at the time of documentation. There are currently 21 tables in the database but these do not represent 21 components as some components encapsulate multiple tables. The design of each table and the choices made are discussed in section \ref{SubSection:Database_Tables}, categorised by components.

There are some consistent attributes across all the tables shown in the diagram. The id column is used as a primary key to provide each tuple with a unique identifier. This primary key is used to manipulate the row based on user input, such as editing and deleting. The last three columns store the date on which the tuple was created, updated, and deleted respectively. The created at column is never updated after the record has been created. The updated at field is changed to the current timestamp whenever the record is changed, e.g. if a user changes their password or the system updates a users reputation. The deleted at column will be null by default, until the user deletes the tuple. This is used for the soft delete feature to prevent it from being selected without actually removing it \cite{PCMEncyclopedia:SoftDelete}.

\begin{figure}[H]
  \centering
  \includegraphics[width=1.0\textwidth]{Images/Design/Database/ERD}
  \caption{Database Schema Represented as an Entity Relationship Diagram (ERD).} \label{fig:Database_ERD}
\end{figure}

\subsection{Tables}
\label{SubSection:Database_Tables}

\subsubsection{User and Authentication}

\subsubsection{Settings}

\subsubsection{Following}

\subsubsection{Categories and Tags}

\subsubsection{Subscriptions}

\subsubsection{Posts and Comments}

\subsubsection{Voting}

\subsubsection{Reporting}

\subsubsection{Notifications}

\subsubsection{Recommendation}

\subsection{Normalisation and Redundancy}
\label{SubSection:Database_Normalisation}

\subsection{Functional Dependencies and Relationship Constraints}
\label{SubSection:Database_Constraints}

\subsection{Security}
\label{SubSection:Database_Security}