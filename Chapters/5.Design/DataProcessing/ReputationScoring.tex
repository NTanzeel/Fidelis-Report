\subsection{Reputation Scoring}
It seems clear that collaboration events are a popular method of assessing reputation in online systems. The Sporas and Histos algorithms demonstrate mechanisms for this scoring using collaboration events \cite{mcnally2013} \cite{zacharia2000}. Ultimately, however, these algorithms did not meet the requirements for reputation scoring in this case. The Fidelis system would, in theory, serve a very large, complex network of users, which constrains the choice of reputation scoring algorithms. The proposed algorithm would need to run on the entire network each day (to score the reputation of both users and posts). Histos and Sporas would both be costly to run given the data stored in our database, therefore a similar, less complex method that adopts elements from Histos and Sporas was considered. This method involves simply performing a sum of each collaboration event, where each event type has a different weighting denoting the relative `importance' of the event in calculating reputation. The weighting functions are shown in equation \ref{eq:rep-weight-post} for reputation scoring posts and equation \ref{eq:rep-weight-user} for reputation scoring users.

\begin{equation}
    \label{eq:rep-weight-post}
     w(c_p) = \left\{\begin{matrix}
            0.1 & if\ c\ is\ an\ upvote\ on\ post\ p \\ 
            -0.1 & if\ c\ is\ a\ downvote\ on\ post\ p \\ 
            0.2 & if\ c\ is\ a\ comment\ on\ post\ p
    \end{matrix}\right.
\end{equation}

\begin{equation}
    \label{eq:rep-weight-user}
        w(c_u) = \left\{\begin{matrix}
            0.3 & if\ c\ is\ an\ upvote\ on\ a\ post\ by\ u\\ 
            -0.3 & if\ c\ is\ a\ downvote\ on\ a\ post\ by\ u \\ 
            0.1 & if\ c\ is\ an\ upvote\ on\ a\ comment\ by\ u \\ 
            -0.1 & if\ c\ is\ a\ downvote\ on\ a\ comment\ by\ u \\ 
            0.2 & if\ c\ is\ a\ comment\ on\ a\ post\ by\ u\\ 
            0.5 & if\ c\ is\ another\ user\ following\ u\\
            -0.5 & if\ c\ is\ another\ user\ reporting\ u
        \end{matrix}\right.
\end{equation}

\noindent
In both cases, the weighting function \(w\) gives the weighting of a given collaboration event \(c\). The values of this function could, of course, be changed at any time in an attempt to produce more accurate reputation scoring. Once a reputation score has been calculated for each user and post, these scores will be normalised between 0 and 1. The pseudocode for this process is shown in algorithms \ref{alg:post-reputation} and \ref{alg:user-reputation} for reputation scoring of posts and users respectively.

\begin{algorithm}[H]
\caption{Post reputation scoring algorithm}
\label{alg:post-reputation}
\begin{algorithmic}[1]
\State $posts\gets getAllPosts()$
\ForAll{posts $p$}
    \State $upvotes\gets w(upvote)\cdot getPostUpvotes(p)$
    \State $downvotes\gets w(downvote)\cdot getPostDownvotes(p)$
    \State $comments\gets w(comment)\cdot getPostComments(p)$
    \State $reputation\gets upvotes + downvotes + comments$
\EndFor
\State $posts\gets scaleReputation(posts)$
\end{algorithmic}
\end{algorithm}

\noindent
Algorithm \ref{alg:post-reputation} begins by retrieving all posts made from the database. For each of these, we then carry out the following steps: for each type of collaboration event, find the number of the collaboration events of that type involving the post and weight this value using the weighting function. The collaboration events considered are: \(upvote\), if the event is an upvote on post \(p\); \(downvote\), if the event is a downvote on post \(p\) and \(comment\), if the event is a comment on post \(p\). These values are then summed and stored as a variable named \(reputation\).

Once a reputation has been calculated for each member of the \(posts\) set, these reputation scores must be normalised between 0 and 1. The formula for this scaling is shown in equation \ref{eq:reputation_scale_post}, where $post_i$ refers to a given post in the set \(posts\), \(max(x)\) is the maximum reputation score of any post in the set \(posts\) and $min(x)$ is the minimum reputation score of any post in the set \(posts\).

\begin{equation}
    \label{eq:reputation_scale_post}
    scaleReputation(post_i) = \frac{post_i - min(posts)}{max(posts) - min(posts)}
\end{equation}

\newcommand{\myindent}[1]{
\newline\makebox[#1cm]{}
}

\begin{algorithm}[H]
\caption{User reputation scoring algorithm}
\label{alg:user-reputation}
\begin{algorithmic}[1]
\State $users\gets getAllUsers()$
\ForAll{users $u$}
    \State $post\_upvotes\gets w(post\_upvote)\cdot getPostUpvotes(u)$
    \State $post\_downvotes\gets w(post\_downvote)\cdot getPostDownvotes(u)$
    \State $comment\_upvotes\gets w(comment\_upvote)\cdot getCommentUpvotes(u)$
    \State $comment\_downvotes\gets w(comment\_downvote)\cdot getCommentDownvotes(u)$
    \State $comments\gets w(comment)\cdot commentsgetPostComments(u)$
    \State $followers\gets w(follower)\cdot getFollowers(u)$
    \State $reports\gets w(report)\cdot getReports(u)$
    \State $reputation\gets post\_upvotes + post\_downvotes + comment\_upvotes + comment\_downvotes \myindent{2.6} + followers$
\EndFor
\State $users\gets scaleReputation(users)$
\end{algorithmic}
\end{algorithm}

\noindent
Algorithm \ref{alg:user-reputation} first retrieves all Fidelis users. For each user we then carry out the following steps: for each type of collaboration event, find the number of the collaboration events of that type involving the user and weight this value using the weighting function. The collaboration events considered are: \(post\_upvote\), if the event is an upvote on any post made by user \(u\); \(post\_downvote\), if the event is a downvote on any post made by user \(u\); \(comment\_upvote\), if the event is an upvote on any comment made by user \(u\); \(comment\_downvote\), if the event is a downvote on any comment made by user \(u\); \(comment\), if the event is a comment on any post made by user \(u\), \(follower\), if the event is another account following user \(u\) and \(reports\), if the event is another account reporting user \(u\). These values are then summed and stored as a variable named \(reputation\).

Once a reputation has been calculated for each member of the \(users\) set, these reputation scores must be normalised between 0 and 1. The formula for this scaling is shown in equation \ref{eq:reputation_scale_user}, where $user_i$ refers to a given user in the set \(users\), \(max(x)\) is the maximum reputation score of any user in the set \(users\) and \(min(x)\) is the minimum reputation score of any user in the set \(users\).

\begin{equation}
    \label{eq:reputation_scale_user}
    scaleReputation(user_i) = \frac{user_i - min(users)}{max(users) - min(users)}
\end{equation}
