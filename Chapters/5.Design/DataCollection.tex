\section{Data Collection}
Fidelis relies heavily on data collected from the users and external sources so that it can successfully provide meaningful content to its users. Because there will be no user data upon launch of the social network, data will need to be collected beforehand so that the functionality is apparent immediately, thus making Fidelis an attractive platform for users to register to. Once users start using Fidelis, data will then need to be continually collected so that the system is able to learn about the users and recommend to them relevant content. In particular, training sets must be collected for the machine learning models used to detect abuse and categorise content and user data must be collected to authenticate access to the site. In addition, when testing the site, template data is used to populate the site to ensure that data is correctly fetched from the database and then displayed. This section discusses the methods which will be used to collect the data so that it can be used in the social network.

\subsection{Training Data}
Both abuse detection and tag categorisation use supervised machine learning techniques in order to classify posts.  Therefore, data must be collected with which the machine learning models can be trained. Because the data will be immutable, storing the training data in Comma-Separated Values (CSV) files will be sufficient.

As part of their competition `Detecting Insults in Social Commentary', Kaggle provided CSV datasets containing sample messages alongside a boolean attribute, representing whether that post is determined to be abusive or not ~\cite{Kaggle:Dataset}. This data is publicly available and the messages are in the same format as Fidelis posts would be expected to be, making it suitable for use when training the model.

Whereas Kaggle have curated a clean dataset which can be readily used in training the abuse detection, a dataset containing sample posts alongside their category was not available. However, Zubiaga and Ji have published a dataset containing Tweet IDs with their corresponding topic tag ~\cite{Zubiaga:Tweets}. For this dataset to be usable in this project when predicting the category of posts on the social network, further data must be collected from the Twitter API so that the text of the Tweet can be determined, since it is the text which contains the features from which the model can learn.

\subsection{Template Data}
Template data is required during implementation of the social network in order to be able to model how data fetched from the database will be reflected throughout the site. Although some of the data will not be used when the site goes live, it is important that the template data closely resembles the format the data is expected to be in when the database is populated with the real data, so that any possible issues with regards to data being collected and displayed can be identified effectively.

Each of the comments, followers, posts, users and votes tables are populated with template data which will be used for testing the site and removed when the site goes live. In addition to this, further template data can be provided in the categories, category\_tag, quotes, tags and wallpapers tables. These tables will remain populated with the data, since the data stored in the tables provide functionality required on the launch of the site. For example, categories is used to store the default categories which are displayed on the Discover page, whereas the quotes and wallpapers tables store the quotes and images displayed on the home page of the site when the user is not logged in.

To ensure that the template data is authentic, the comments and posts are obtained either from existing social networks such as Facebook and Twitter or they will be suggested by would-be users of Fidelis. In addition, these would-be users can be added as template users to the site.

\subsection{User Authentication}
Data will need to be collected about the user in order to authenticate access to private areas within the site. In Fidelis, this will use a form which is filled in by the user during registration, which includes password fields which will grant the user access to the site in the future. This password will then be encrypted and stored in the users table. The user's email address will also be collected and stored so that the password can be recovered in the case that the user forgets their password. This is consistent with the methods used to collect authentication data in current social networks.