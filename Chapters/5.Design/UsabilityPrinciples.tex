\section{Usability Principles}
Much progress has been made in the design of applications, making it more enjoyable for users to interact with a site and therefore increasing the likelihood of them recommending and revisiting the site. David Benyon composed usability principles which act as a guideline when designing applications with the user in mind \cite{Benyon}. These principles aim to improve consistency, familiarity and intuitiveness of applications.

The screenshots in figure \ref{fig:Twitter_Changes} highlight how the design of websites have changed. In particular, the design in image \ref{fig:Twitter_2017} shows how Twitter now use a more simplistic flat design, which uses bright colours so key areas of the page are clearly visible and separated. This is as opposed to the design in \ref{fig:Twitter_2006}, which uses a lot of gradients. This draws attention away from the important features on the site and makes it appear more cluttered. This shows how the designs have been adapted in order to make them more usable. Fidelis will also aim to make the application more usable through the use of design principles, as discussed in this section.

\begin{figure}[H]
	\centering
	\begin{subfigure}[t]{0.45\textwidth}
		\centering
		\includegraphics[width=1.0\textwidth, height=125px]{Images/Design/Twitter_2006}
		\caption{Twitter in 2006}\label{fig:Twitter_2006}		
	\end{subfigure}
	\quad
	\begin{subfigure}[t]{0.45\textwidth}
		\centering
		\includegraphics[width=1.0\textwidth, height=125px]{Images/Design/Twitter_2017}
		\caption{Twitter in 2017}\label{fig:Twitter_2017}
	\end{subfigure}
	\caption{Twitter in 2006 and 2017}\label{fig:Twitter_Changes}
\end{figure}

\subsection{Consistency}
By ensuring that the design of the application is consistent across all pages in the site, users will be able to learn quickly how to navigate and use features of the site. An example of this is the navigation bar, which will be included on all of the pages, allowing the user to search and log in and out. Figure \ref{fig:navs} demonstrates how Facebook and Twitter have designed a similar navigation bar which features throughout their site. Also, Fidelis will make use of widgets, which can be reproduced on multiple pages across the site. This allows features to be repeated, without having to change style. Doing so allows the users to understand the possible actions they are able to make on each page and improves ease of use.

\subsection{Familiarity}
Whilst aiming to differentiate Fidelis from other social networks by implementing features which make it more trust-centric, keeping a familiar design will allow users to use the site with minimum learning. Therefore, language used on the site will be similar to the language in pre-existing social networks. This includes the terms `following' and `followers' to describe who a user is connected to within the network and `trending' to show the most common terms which are being mentioned within a period of time. In addition, # will be used to tag a post and \@ will be use to mention a user in a post, as is the case with social networks such as Twitter, Instagram and Facebook. Other features such as profile picture, wallpapers and notifications will also follow a familiar format in comparison with current popular websites.

\subsection{Intuitive Design}
By using an intuitive design, the usability of Fidelis will increase, as it would become easier and quicker for users to complete tasks. For example, affordances could be used so that the purpose of a feature becomes obvious to the user through the design only. This could include using a magnifying glass to symbolise search or using a dust bin for deleting an item. There is also additional functionality which can be added to the site to make navigation more intuitive, such as clicking on the logo to redirect to the home page. Furthermore, displaying search suggestions as a user types in the search bar allows the user to quickly navigate to the desired page without having to redirect to a search results page. Not only this, but showing search suggestions aids ease of use, such as when the user is unsure of spelling. Figure \ref{fig:fb-search} shows how Facebook have designed their search alongside suggestions.

\begin{figure}[H]
	\centering
	\includegraphics[height=125px]{Images/Design/fb-search}
	\caption{Facebook search}\label{fig:fb-search}		
\end{figure}