\chapter{System Requirements}
\label{Chapter:SystemRequirements}

\section{Requirement Refinement}

\section{Functional Requirements}
The functional requirements will guide the development of all aspects of the system. These requirements will ensure that the system provides the necessary functionality, the implementation details of which are entirely up to the developer and may change during the development process.
\begin{enumerate}[label=\textbf{F\arabic*}]
	\item The system must be able to communicate with a number of third party APIs in order to retrieve data.
		\begin{enumerate}
			\item The system must be able to convert all data received from third parties into a consistent format.
		\end{enumerate}
	\item Users must be able to register and log into the system.
	\begin{enumerate}
		\item Registration may be done through manually entering all the required details or by using a third party which provide oAuth 2.0 services, such as but not limited to Facebook and Twitter.
		\item These credentials should be stored, in an encrypted format, so that they can later be used for authentication.
		\item Users should be able to recover their account in the case that they have forgotten their password or their account has been compromised.
		\item A user maintains the right to be able to delete their account publicly, however this data may still be retained privately.
	\end{enumerate}
	\item A user may make their account private, preventing other users from being able to see the content they share.
		\begin{enumerate}
			\item If a user attempts to follow a private account, a request to follow is sent to the private user and must be approved before the relationship is created.
		\end{enumerate}
	\item A user may block another user.
		\begin{enumerate}
			\item If the blocked user is already following the blocking user, then they automatically unfollow the user.
			\item Once a user has been blocked, the blocked user should not be able to find the blocking users profile through search or otherwise.
			\item If a user has blocked a user then they may also unblock the user. This will not restore any previous following relationships.
		\end{enumerate}
	\item In order to see content, users should be able to add other users to their ��trust circle��.
		\begin{enumerate}
			\item This is identified as a one way relationship in which ��user A follows user B�� implies that ��user A trust user B��.
			\item Similarly, if users no longer wish to see content from specific users then they may unfollow a user they are already following.
		\end{enumerate}
	\item The system will overcome the cold-start problem, under which a user will not see anything on their timeline upon registration, by providing a set of predefined categories which can be used to explore.
		\begin{enumerate}
			\item Users are able to access a customised `Discover' feed by subscribing to categories, sub categories or topics.
			\item The `Discover' feed will then be generated by pulling content from the various subscriptions allowing the user to explore topics based on their interest, discover new content as well as new users who they may follow.
		\end{enumerate}
	\item The system will provide a personal feed where the user is able to view content from the people they trust, prioritised by the reputation of the people they follow as well as the reputation of the content itself.
		\begin{enumerate}
			\item The user may hide content from their personal feed which they find offensive, or just uninteresting. This should be used to learn and adapt the recommendation system by modifying the posters reputation.
		\end{enumerate}
	\item A user must be able to post new content onto the system.
		\begin{enumerate}
			\item A post may include text, images or videos. Additionally an image or video post may also contain text.
			\item When posting content, users may optionally tag the post or leave it untagged.
			\item When the user tags a post with a popular topic, e.g. \#Brexit, the post should automatically be assigned to the correct category, e.g. Politics in the aforementioned case, using a bucket of keywords per category.
			\item If a tag is mentioned and the post cannot be categorised automatically then the user may be prompted to assign one of the predefined categories so that the system can learn and adapt for the future. Additional processes may be used to verify that tags are correctly classified.
		\end{enumerate}
	\item Users can view any text, photo or video posts made by public accounts or by private accounts which they follow.
		\begin{enumerate}
			\item Viewing media content should open up in a popup or similar so users can view the post and comment on it.
		\end{enumerate}
	\item Users must be able to reply to a post made by another user.
		\begin{enumerate}
			\item If the user has made their account private then only users following the author of the post may comment on or see the post.
		\end{enumerate}
	\item Users may interact with content they come across by liking, disliking or sharing it. Each of these actions will impact the reputation of the content.
		\begin{enumerate}
			\item The user should also be able to reverse their interaction, e.g. un-liking a liked post.
		\end{enumerate}
	\item The system will automatically calculate a reputation score for users which symbolises the trustworthiness of the user and the content they post.
		\begin{enumerate}
			\item This score will be impacted by a range of factors. For example, a user'��s trustworthiness increases as their number of followers (people who trust the user) goes up. Similarly, this reputation can also decrease if the users content is disliked.
			\item This will be used, along with other factors such as mutual connections, to recommend other users the user may be interested in following.
		\end{enumerate}
	\item The system will also calculate a reputation for each user posted content to represent the quality of content.
		\begin{enumerate}
			\item As other users react to the content by liking, disliking or sharing it, the reputation of the post will change which in turn impacts the author's reputation.
			\item Highly scored posts will be more likely to be recommended to other users on the `Discover' or `Categories' pages.
		\end{enumerate}
	\item Each user will be able to choose how the reputation is calculated and how the notion of trust is implemented.
		\begin{enumerate}
			\item This feature will be available through the settings menu. Each algorithm will use different properties of a user and posts to compute the score.
			\item Additionally, the user will be able to change the sensitivity of the scoring system from mild to strong.
		\end{enumerate}
	\item The system will incorporate abuse detection to prevent things such as profanity, swearing and nudity amongst other things.
		\begin{enumerate}
			\item Each user can adjust the sensitivity of their abuse detection system depending on which content they would like to hide.
			\item Due to the processing requirements, this may be implemented as a job which executes at regular interval to remove content.
		\end{enumerate}
	\item Users can report content which they find is inappropriate or offensive, such as profanity or nudity.
		\begin{enumerate}
			\item The report along with the reported content is sent to a moderation system. This report must be picked up by a moderator who either dismisses the reported if the content adheres to the guidelines or the content is removed and the user's reputation is changed to reflect this.
		\end{enumerate}
\end{enumerate}

\section{Non-Functional Requirements}
In order for the system to be successful and used, it is essential that it is available on demand, regardless of time and place. This means that the system must support a range of devices varying from desktop to small hand-held smartphones. The non-functional requirements for this project will not only help to ensure that the systems operates smoothly and provides a responsive user experience, but also ensure that the development of the system is up to a high standard.
\begin{enumerate}[label=\textbf{NF\arabic*}]
	\item \textit{Compatibility}: The system must be cross browser compatible and support all devices with a modern browser.
		\begin{enumerate}
			\item A responsive design and structure must be used to ensure the system is compatible with all devices.
			\item The following categories of devices must be well supported.
			\begin{enumerate}
				\item Mobile
				\item Tablet
				\item Desktop
			\end{enumerate}
			\item Functionality should not be limited or restricted on any of the given devices but may be implemented differently
		\end{enumerate}
	\item \textit{Usability}: The system must be intuitive and user friendly.
		\begin{enumerate}
			\item The system must be intuitive and easy to navigate. Users must be able to access pages without any guidance.
			\item The user experience across all pages must be consistent in terms of design and functionality.
			\item The user must never encounter errors, but if the system encounters an error then an appropriate output should be produced.
			\item Appropriate user feedback must be given when interacting with the system.
			\item The system must have an appropriate load time across all pages.
			\begin{enumerate}
				\item Studies have shown that nearly half of the users consider abandoning a site if it takes longer than 3 seconds to load ~\cite{Kissmetrics:Speed}.
			\end{enumerate}
		\end{enumerate}
	\item \textit{Security}: The system and the data held must be secure.
		\begin{enumerate}
			\item The users details and credentials, such as email and password must be appropriately secured.
			\item Any data added by the user or associated to the user must only be available to the user unless explicitly made available to the public.
			\item Users may not have access to personal information about other users, such as email address and location data.
			\item Any resources uploaded by the user must not be browsable by other users, unless made public by the user.
		\end{enumerate}
	\item \textit{Scalability}: The system must be scalable and respond well to growth.
		\begin{enumerate}
			\item Data storage and processing decisions must be made whilst taking growth and expansion into consideration.
			\item Growth should not limit functionality and availability - the system must be able to cope with this.
			\item As storage and processing can be costly, these must be considered when developing the system so efficient use is made of both of these resources.
		\end{enumerate}
	\item \textit{Extensibility}: The system must be extensible and support further development.
		\begin{enumerate}
			\item The system must be designed modularly, regardless of whether it may need changing or not.
			\item Design decisions must be made with extensibility and growth in mind and as a result, any component or part of the system must be easily replaceable with an upgrade.
			\item Each of these components must be independent and operate independently.
		\end{enumerate}
	\item \textit{Maintainability}: The system must be maintainable.
		\begin{enumerate}
			\item Standard and good code practices should be adhered to whilst developing the system.
			\item A version control system should be used to make regular checkpoints.
			\item The system and any external resources or technologies should always be kept up-to-date to avoid any issues.
			\item Any hardware decision made must take maintainability into consideration.
		\end{enumerate}
	\item \textit{Readability}: The system must be well documented
		\begin{enumerate}
			\item The codebase must be thoroughly documented using any standard commenting conventions and guidelines for the specific language.
			\item Any progress logs must provide a clear outline of the progress and changes made.
			\item Version control commits must detail the changes made - any addition or removal of features must be clearly stated.
			\item A clear outline of the system and its functionality must be provided.
		\end{enumerate}
\end{enumerate}

\section{Limitations and Constraints}