\chapter{Legal, Social, Ethical and Professional Issues}
\label{Chapter:Issues}

\section{Legal Issues}
\subsection{Sensitive Data}
\subsection{Licensing}
\subsection{Resources}

\section{Social Issues}
Despite one of the main features of Fidelis being abuse detection, it is important to consider the implications should certain abusive posts go undetected. Due to the nature of social media and the difficulty to systematically detect the semantics of a post, it is likely that there will be posts which are not detected by the abuse algorithms. As a result, functions such as being able to block a user and delete posts must be included, in the event a user does receive abusive content.

As well as abusive content, further offensive content may be posted on the site which does not fall under the category of abuse. This includes nudity and links to inappropriate or illegal content. As with abusive content, the user will be able to remove content from their feed which they find offensive, but in extreme cases users may have to be reported to the authorities or removed from the Fidelis site altogether.

In addition to abusive content, Fidelis would also be subject to other social issues which have been recognised with preexisting social networks. This includes possible mental health implications of socialising online instead of face-to-face. Research has suggested that ``digital communications less able to lower depression risk'' in comparison with ``people who regularly met in person with family and friends''~\cite{OHSU:Depression}. Therefore, it is important that Fidelis attempts to mitigate this risk by encouraging users not to spend too much time on the social network in its `Terms of Use' and also offers support by providing contact details of relevant helplines for those who seek help.

\section{Ethical Issues}

\section{Professional Issues}
For the Fidelis to be able to function, it will be required to store large amounts of user data. It is therefore necessary to ensure that the privacy of the users is protected, by abiding by the `Data Protection Act' of 1998. This legislates that data is ``fairly and lawfully processed; processed for limited purposes and not in any manner incompatible with those purposes;
adequate, relevant and not excessive; accurate and where necessary, up to date; not kept for longer than is necessary; processed in line with the data subject's rights; secure and that personal information shall not be transferred to countries outside the EEA without adequate
Protection''~\cite{DPA}. So that the Fidelis system meets these requirements it is paramount that measures are taken to secure the database and to minimise corruption to the data, as well as regularly maintaining the data to remove anything which is surplus to the requirements of the system.