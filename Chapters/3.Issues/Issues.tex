\chapter{Legal, Social, Ethical and Professional Issues}
\label{Chapter:Issues}
As with any system involving human interaction, there are a number of legal, ethical, social and professional issues that should be considered. Due to the scope of this project and the sheer amount of human interaction involved, it is critical to be aware of the possible issues that could occur and find ways to address them.

\section{Legal Issues}
It is of vital importance that actions are taken to conform as closely as possible to both local and international laws. It is important to be completely clear with all stakeholders what is expected of them and what the terms of service, policies and procedures for the platform entail.

\subsection{Sensitive Data}
Since the proposed system focuses on users creating and sharing content, it is difficult to control what users will post. The platform is intended to promote an environment for free speech and sharing opinions, however there may be cases where a user publicly shares some information which was intended to be private or confidential. This could be personal information, or information pertaining to a third party such as an employer. Once information has been shared in this way there is little recourse; as soon as it is public, anyone can view it. In such a case one may argue that since the information was posted and shared through the Fidelis platform, the platform rather than the user is liable. It is therefore important to make it clear through policies and procedures that users are responsible for any content that they post or share through this platform. In a similar way, the use of unauthorised third party trademarks or copyright-protected works may be used by an individual in a way that infringes on the trademark or copyright. Users should therefore also be made aware that they are liable for any such infringement and that their posts may be removed because of this.

\subsection{Licensing}
Just as third-party materials posted to a social media site may infringe copyright or trademarks, posting photographs and video without proper releases may violate the privacy or publicity rights of individuals. Additionally, within certain industries, employees must ensure that they do not violate specific privacy regulations of their employer in their activities on social media sites. This phenomena should, however, be combatted by the combination of reputation-based content filtering and the ability for users to report content.

\subsection{Resources}
In addition to ensuring users respect and uphold copyright and trademarks, it is important that the platform itself does not make use of any unlicensed materials. This includes, but is not limited to, using copyright-free images and ensuring that the required permissions are fulfilled for any software used in the creation and running of the platform. These guidelines were clearly relayed to all project team members to ensure no legal violations occurred.

\subsection{User Protection}
For all users of the Fidelis platform there is a responsibility to ensure that they are being treated fairly and legally. A common trend across many social media platforms is targeted `abuse' towards a specific individual. This can be in the form of `cyber bullying', `trolling’ or in some extreme cases, defamation. Defamation is defined as ``A false statement or fact, not made under privilege, that is communicated to a third person and that causes damage to a person’s reputation. For public figures, the plaintiff must also prove actual malice.'' \cite{BusinessLawToday}. Fidelis is focused on filtering the content that is presented to a user, which is intended to reduce the amount of negative content the user will see. While very negative content will be hidden from many users, the posts will still persist on the platform and so any content considered defamatory is still subject to defamation law. It should therefore be possible to delete these posts if needed as well as possibly using the posts to find the IP address or some other information about the post author (i.e. to aid authorities whilst following both the law and any policies regarding privacy or anonymity of the user).

\subsection{International Law}
One particular legal pitfall to consider is local versus international law. Social networks often have users posting from many different countries all over the world. A social media platform may adhere to all pertinent local laws in the country it was created in, however complications may arise due to unforeseen and possibly conflicting laws if the platform is operating in another country. With a vast number of legal jurisdictions, it can be difficult to keep up with international law. However there are actions that can be undertaken to maintain a strong legal position. For example, serving only select countries by limiting the selection of language options, or even by restricting the IP addresses of visitors to only allow users from specific countries \cite{Olswang}.

\section{Social Issues}
Despite one of the main features of Fidelis being abuse detection, it is important to consider the implications should certain abusive posts go undetected. Due to the nature of social media and the difficulty in systematically identifying the semantics of a post, it is likely that there will be posts which are not detected by the abuse algorithms. As a result, functions such as being able to block a user and delete posts must be included, in the event a user does receive abusive content.

As well as abusive content, further offensive content may be posted on the site which does not fall under the category of abuse. This includes nudity and links to inappropriate or illegal content. As with abusive content, the user should be able to remove content from their feed which they find offensive, but in extreme cases users may have to be reported to the authorities or removed from the Fidelis site altogether.

In addition to abusive content, Fidelis would also be subject to other social issues which have been recognised with pre-existing social networks. This includes possible mental health implications of socialising online instead of face-to-face. Research has suggested that ``digital communications less able to lower depression risk'' in comparison with ``people who regularly met in person with family and friends''~\cite{OHSU:Depression}. Therefore, it is important that Fidelis attempts to mitigate this risk by encouraging users not to spend too much time on the social network in its `Terms of Use' and also offers support by providing contact details of relevant helplines for those who seek help.

\section{Ethical Issues}
As a social media platform, there is a level of ethical responsibility to protect users by ensuring an agreed upon level of privacy. For example it is fairly common for company Human Resources departments to review the social media pages of both job candidates and current employees. While this practice may be of use to the company, the design of the Fidelis platform is such that users should feel comfortable posting and discussing content without feeling the need to censor themselves. The user may not wish for everybody to see their posts and so as a compromise the user will be able to make their account private, preventing non-approved users from seeing the content they share.

This system aims to provide a platform for users to post and share content relating to a wide range of topics. In some cases, for example with politics, there are often issues that see people taking very different viewpoints. To provide a balanced platform for conversation and debate on these topics, it is crucial to take an unbiased approach to how the system decides which content to show to a user and which users to recommend to each other. It would be very easy for a person designing such a system to be biased towards their own opinions, presenting all users with the same content that provides a one-sided argument. This would defeat the point of the Fidelis platform and be taking advantage of the user’s trust in the system, so clearly these tactics will not be employed. All content suggestion will be focus on user preference only.

\section{Professional Issues}
For the Fidelis to be able to function, it will be required to store large amounts of user data. It is therefore necessary to ensure that user privacy is protected maintained. This will be achieved by abiding to the `Data Protection Act' of 1998 ~\cite{DPA}. This legislates that data is:

\begin{enumerate}
\item Fairly and lawfully processed; 
\item Processed for limited purposes and not in any manner incompatible with those purposes
\item Adequate, relevant and not excessive
\item Accurate and where necessary, up to date
\item Not kept for longer than is necessary
\item Processed in line with the data subject's rights 
\item Appropriate technical and organisational measures shall be taken against unauthorised or unlawful processing of personal data and against accidental loss or destruction of, or damage to, personal data
\item Secure and that personal information shall not be transferred to countries outside the EEA without adequate Protection
\end{enumerate}

So that the Fidelis system meets these requirements it is paramount that measures are taken to secure the database and to minimise corruption to the data, as well as regularly maintaining the data to remove anything which is surplus to the requirements of the system.