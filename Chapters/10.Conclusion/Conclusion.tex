\chapter{Conclusion}
\label{Chapter:Conclusion}
Despite the significant progress made throughout the duration of the project, additional improvements may be made in the future, in order to further Fidelis as a trust-centric social network. In this section, the progress of the project is reflected on. In addition, the work which could be implemented in the future is discussed, which would allow the platform to continue to improve beyond the initial scope of this project.

\section{Summary}
The aim of this project was to provide a trust-based social network for user engagement and protection. The product realised at the end of this project provides a fully-functional social network. Not only this, but the delivered product also achieves its goal of being a trust-based social network. Through the use of abuse detection, user reputations and content filtering, user protection from abusive content or content they do not find interesting is ensured. With content recommendation, users can be sure that they will be provided with content and other users they will enjoy engaging with. This, along with meeting all the specified functional and non-functional requirements ensured that stakeholder satisfaction was guaranteed. 

The final deliverable goes beyond being just a minimum viable product and instead exceeds the expectations set out at the beginning of the project.

\section{Future Work}
Balancing development work with academic commitments meant that amount of time dedicated to developing Fidelis was limited. As a result of this, although all functional requirements were met, more can still be done to improve Fidelis and take it beyond where it already is. To this end, a discussion will follow, looking at some of the ways the system can be improved upon.

\subsection{Improved User Interface}
Currently, the Fidelis UI is near identical to that of Twitter. This was a sentiment both applauded and criticised by users throughout UAT, as exemplified in the testimonials given by Vincent and Greenwood. A large part of future work should, therefore, focus on improving the current UI design. Although the design should be re-worked, users did enjoy the familiarity of design with Twitter so some aspects of the UI should be retained. What is retained and changed is left at the discretion of future developers.

\subsection{More Complex Reputation Scoring}
The reputation scoring system currently implemented is very simple. To provide users with up-to-date reputations, it was decided to implement a computationally fast algorithm. The trade-off as a result of this decision means that some might deem the complexity of scoring insufficient. However, the voting system was chosen, although simple has been proven to be effective given its use by Reddit and Youtube \cite{Reddit:About, Youtube:Home}. Nonetheless, this system can be prone to bias, which we see with likes on YouTube. A video can be of good quality, but if users do not like the author of the video e.g. most videos related to Justin Bieber, votes won't reflect the quality of the content. The weighting system used avoids this to an extent, but more complex techniques can be used. Future work should, therefore, involve researching algorithms that can perform quick reputation scoring, whilst using more complex techniques that will only improve upon current scoring mechanisms.

\subsection{Classifier Refinement}
The data used for model training and evaluation was collected from Twitter and other resources. As a result of this, the size of the dataset was limited and dated. To provide better performing models, classifiers should be re-trained with new user data once the network becomes more active. By re-evaluating models with current data, the dataset used for training will be larger, leading to potential performance improvements. Given more time, model fine-tuning should take place with more in-depth parameter searching. This could again improve model performance. Another refinement which could be made to tag categorisation model is to train this model on individual tags rather than entire posts, in order to improve the performance of the system in this area. This would require a large enough training set of tag-category pairs with which a model would be able to learn.

\subsection{News Recommendations}
One of core functionalities discussed throughout the report has been providing users with recommendations. This functionality was successfully implemented, but the system should not limit itself to just the current forms of recommendations. An idea that was discussed during development was to provide users with news articles relevant to what they post.
This piece of work would involve attaching news stories to a users' post that are related to the topic of the post. Facebook currently provides trending news stories, but these articles may not always be of interest to the user. The driving factor behind this concept is to again provide engaging content to the user. This work would also help in reducing the echo chamber effects as articles provided would only be related to the topic of the user's posts, and not necessarily reflect their own sentiment.

\subsection{Additional Social Network Functionality}
Existing social networks provide a standard set of ``expected'' functionality. Common things included in this are sharing posts, posting videos and user messaging services. The aim of Fidelis was to tackle the downfalls of existing platforms by providing a novel solution addressing these issues, but as the focus was on this the final product failed to implement the aforementioned functionalities. Therefore, any future work on the system should include integration of these functionalities. Providing these functionalities would broaden channels through which users can engage with content and each other.